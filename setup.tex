\usepackage[UTF8]{ctex}
\usepackage[a4paper, top=2cm, bottom=2cm, left=2.5cm, right=2.5cm]{geometry}
\usepackage{listings}
\usepackage{titlesec}
\usepackage{datetime}
\usepackage{graphicx}
\usepackage{tikz}
\usepackage{xpatch}
\usepackage{titling}
\usepackage{makeidx}
\usepackage{tocloft}
\usepackage{hyperref}
\usepackage{fancyhdr}
\usepackage{lastpage}
\usepackage{lipsum}
\usepackage{tcolorbox}
\usepackage{array, booktabs, tabularx}
\usepackage[table]{xcolor}
\usepackage{enumitem}
\usepackage{xeCJK}
\usepackage{amsmath, amssymb, amsthm}
\tcbuselibrary{theorems, breakable, skins}



\graphicspath{{./figure/}}

% 目录页眉
\fancypagestyle{plain}{
    \fancyhf{}
    \lhead{\\\
    \scriptsize{2025~年~05~月}}
    \chead{\centering{算法竞赛模板 \\
    \scriptsize{\textbf{The template of Competitive Program by mnnuer}}}}
    \rhead{\\
    \scriptsize{May, 2025}}
}


% 定义章节页眉显示规则
\fancypagestyle{mainstyle}{
    \fancyhf{}
    \fancyhead[L]{\small\itshape\leftmark}  % 左侧显示当前章节
    \fancyhead[R]{\small\thepage}           % 右侧显示页码
    \renewcommand{\headrulewidth}{0.4pt}
}




% 封面设计
\newcommand{\coverpage} {
    \thispagestyle{empty}
    \begin{center}
        \vspace*{5cm}
        % \includegraphics[width=0.3\textwidth]{icpc.jpg} \\[1cm]

        % 使用两个minipage实现并排图片
        \begin{minipage}[c]{0.48\textwidth} % 左边图片
            \centering
            \includegraphics[width=0.9\linewidth]{闽南师范大学.jpg}
        \end{minipage}
        \hfill % 添加弹性间距
        \begin{minipage}[c]{0.48\textwidth} % 右边图片
            \centering
            \includegraphics[width=0.9\linewidth]{icpc.jpg}
        \end{minipage}
        \\[1.5cm] % 图片下方间距
        \textbf{\Huge 算法竞赛模板} \\[1cm]
        

        \begin{tabularx}{0.8\textwidth}{Xl}
            组织编写: & 22网络-1 肖建华 \\
            内容编写: & 22网络-1 肖建华 \\
            最后提交日期: & \today \\ 
        \end{tabularx} \\[2cm]
        
        \textit{\large 计算机学院} \\
        \textcolor{blue}{\large 2024年春季学期}


    \end{center}
    \clearpage
}

\tcbset{
    outstyle/.style={ % 自定义样式名称(可任意命名)
        colback = white,          % 背景色
        colframe = black,         % 边框色
        arc = 3mm,                % 边框圆角
        boxrule = 1pt,            % 边框粗细
        fonttitle = \bfseries,
        left = 5mm,               % 左内边距
        right = 5mm,              % 右内边距
        top = 2mm,                % 上内边距
        bottom = 2mm,             % 下内边距
    },
}


% ========== 亮色主题代码样式 ==========
\definecolor{lightBG}{RGB}{255,255,255}       % 白色背景
\definecolor{lightText}{RGB}{36,41,46}        % 深灰文本
\definecolor{lightKeyword}{RGB}{215,58,73}    % 红色关键字
\definecolor{lightString}{RGB}{3,47,98}       % 深蓝字符串
\definecolor{lightComment}{RGB}{106,115,125}  % 灰绿注释

\lstset{
    backgroundcolor=\color{lightBG},       % 白色背景
    basicstyle=\ttfamily\color{lightText}, % 深灰文本
    keywordstyle=\color{lightKeyword},     % 红色关键字
    stringstyle=\color{lightString},       % 深蓝字符串
    commentstyle=\itshape\color{lightComment}, % 灰绿注释
    breaklines=true,
    numbers=left,
    numberstyle=\large\color{gray},      % 浅灰行号
    framesep=10pt,
    tabsize=4,
    showspaces=false,
    showstringspaces=false,
    language=C++,
}


% 定义表格背景颜色
\definecolor{headerbg}{RGB}{230, 240, 255}
\definecolor{gray10}{RGB}{240, 240, 240}

% 定义列表
\renewcommand{\labelenumii}{\arabic{enumi}.\arabic{enumii}}
\renewcommand{\labelenumiii}{\arabic{enumi}.\arabic{enumii}.\arabic{enumiii}}
\renewcommand{\labelenumiv}{\arabic{enumi}.\arabic{enumii}.\arabic{enumiii}.\arabic{enumiv}}


% 定义标题
\renewcommand{\cfttoctitlefont}{\Huge\bfseries\centering}
\setlength{\cftbeforetoctitleskip}{-2em} % 标题位置调整

% 定义定义,理论等
\newtheorem{theorem}{定理}[section]
\newtheorem{definition}{定义}[section]
\newtheorem{corollary}{推论}[section]
\newtheorem{lemma}{引理}[section]

% ========== 目录格式设置 ==========
\titleformat{\section}[block]{\Large\bfseries\filcenter}{}{0em}{}

% 设置四级标题
\setcounter{secnumdepth}{4}
\setcounter{tocdepth}{4}

