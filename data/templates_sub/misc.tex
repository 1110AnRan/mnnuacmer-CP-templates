\section{算法杂项专题模板}
\subsection{FastIO}
\begin{lstlisting}
namespace io_lib {
#ifdef FREAD
#define MAXBUFFERSIZE 1000000
inline char fgetc() {
  static char buf[MAXBUFFERSIZE + 5], *p1 = buf, *p2 = buf;
  return p1 == p2 && (p2 = (p1 = buf) + fread(buf, 1, MAXBUFFERSIZE, stdin), p1 == p2) ? EOF : *p1++;
}
#undef MAXBUFFERSIZE
#define getchar fgetc
#endif
#define gc getchar
struct IOReader {
  template <typename T, typename std::enable_if<std::is_integral<T>::value, int>::type = 0>
  inline const IOReader& operator>>(T& a) const {
    a = 0;
    bool flg = false;
    char ch = gc();
    while (ch < '0' || ch > '9') {
      if (ch == '-') flg ^= 1;
      ch = gc();
    }
    while (ch >= '0' && ch <= '9') {
      a = (a << 3) + (a << 1) + (ch ^ '0');
      ch = gc();
    }
    if (flg) a = -a;
    return *this;
  }
  inline const IOReader& operator>>(std::string& a) const {
    a.clear();
    char ch = gc();
    while (isspace(ch) && ch != EOF) ch = gc();
    while (!isspace(ch) && ch != EOF) a += ch, ch = gc();
    return *this;
  }
  inline const IOReader& operator>>(char* a) const {
    char ch = gc();
    while (isspace(ch) && ch != EOF) ch = gc();
    while (!isspace(ch) && ch != EOF) *(a++) = ch, ch = gc();
    *a = '\0';
    return *this;
  }
  inline const IOReader& operator>>(char& a) const {
    a = gc();
    while (isspace(a)) a = gc();
    return *this;
  }
  template <typename T, typename std::enable_if<std::is_floating_point<T>::value, int>::type = 0>
  inline const IOReader& operator>>(T& a) const {
    a = 0;
    bool flg = false;
    char ch = gc();
    while ((ch < '0' || ch > '9') && ch != '.') {
      if (ch == '-') flg ^= 1;
      ch = gc();
    }
    while (ch >= '0' && ch <= '9') {
      a = a * 10 + (ch ^ '0');
      ch = gc();
    }
    if (ch == '.') {
      ch = gc();
      T p = 0.1;
      while (ch >= '0' && ch <= '9') {
        a += p * (ch ^ '0');
        ch = gc();
        p *= 0.1;
      }
    }
    if (flg) a = -a;
    return *this;
  }
  template <typename T1, typename T2>
  inline const IOReader& operator>>(std::pair<T1, T2>& p) const {
    return operator>>(p.first), operator>>(p.second), *this;
  }
  template <typename T, const unsigned long long N>
  inline const IOReader& operator>>(std::array<T, N>& p) const {
    for (unsigned long long i = 0; i < N; i ++)
      operator>>(p[i]);
    return *this;
  }
  template <typename... Ts>
  inline const IOReader& operator>>(std::tuple<Ts...>& p) const;
#undef importRealReader
};
const IOReader io;
#undef gc
template <typename T>
void read(T& val) { io >> val; }
template <typename T>
void read(int l, int r, T& A) { for (int i = l; i <= r; i++) io >> A[i]; }
template <typename T>
void write(const T& A, int l, int r, const char* sp, const char* end = "") { for (int i = l; i <= r; i++) printf(sp, A[i]); printf("%s", end); }
template <typename T>
void write(const auto& A, const T* sp, const char* end = "") {for (auto e : A) printf(sp, e); printf("%s", end); }
template <typename T = int>
T read() { T res; io >> res; return res; }
template <typename T, int N>
std::array<T, N> read() { return read<std::array<T, N>>(); }
template <typename Tuple, typename Func, size_t... N>
void func_call_tuple(Tuple& t, Func&& func, std::index_sequence<N...>) { static_cast<void>(std::initializer_list<int>{(func(std::get<N>(t)), 0)...}); }
template <typename... Args, typename Func>
void travel_tuple(std::tuple<Args...>& t, Func&& func) { func_call_tuple(t, std::forward<Func>(func), std::make_index_sequence<sizeof...(Args)>{}); }
template <typename... Ts>
std::tuple<Ts...> reads() {
  std::tuple<Ts...> res;
  travel_tuple(res, [&](auto&& val) { io >> val; });
  return res;
}
template <typename... Ts>
inline const IOReader& IOReader::operator>>(std::tuple<Ts...>& p) const { return p = reads<Ts...>(), *this; }
template <typename T = int>
std::vector<T> getv(int n, int start = 0) {
  std::vector<T> res(start + n);
  for (int i = start; i < start + n; i++) io >> res[i];
  return res;
}
template <typename T, typename T1, typename... Ts>
std::vector<std::tuple<T, T1, Ts...>> getv(int n, int start = 0) {
  std::vector<std::tuple<T, T1, Ts...>> res(start + n);
  for (int i = start; i < start + n; i++) io >> res[i];
  return res;
}}  // namespace io_lib
using namespace io_lib;

#define cin io
\end{lstlisting}

\subsection{defs}
\begin{lstlisting}
namespace defs {
#define YES cout << "YES" << endl;
#define NO cout << "NO" << endl;
#define Yes cout << "Yes" << endl;
#define No cout << "No" << endl;
#define all(x) (x).begin(), (x).end()
#define rall(x) (x).rbegin(), (x).rend()
#define rep(i, j, k) for(int i = (j); i <= k; ++i)
#define per(i, j, k) for(int i = (j); i >= k; --i)
#define multiCase()     \
    int totCases; std::cin >> totCases; \
    for(int currCase = 1; currCase <= totCases; currCase++)
using i32 = int;
using u32 = unsigned int;
using i64 = long long;
using u64 = unsigned long long;
using i128 = __int128;
using u128 = __uint128_t;
using f32 = float;
using f64 = double;
using TII = std::tuple<int, int, int>;
const i64 mod = 1'000'000'007 /* 998'244'353 */;
template <typename T> void sort(T& v) { std::sort(all(v)); }
template <typename T> T sorted(T v) { return std::sort(v), v; }
template <typename T> void rsort(T& v) { std::sort(rall(v)); }
template <typename T, typename T2> void sort(T& v, T2 compare) { std::sort(all(v), compare); }
template <typename T, typename T2> T sorted(T v, T2 compare) { return std::sort(v, compare), v; }
template <typename T> void reverse(T& v) { std::reverse(all(v)); }
template <typename T> T reversed(T v) { return std::reverse(v), v; }
template <typename T> void unique(vector<T>& v) { v.erase(std::unique(all(v)), v.end()); }
template <typename T> vector<T> uniqued(vector<T> v) { return std::unique(v), v; }
template <typename T> T min(const vector<T> &v) { return *std::min_element(all(v)); }
template <typename T> T max(const vector<T> &v) { return *std::max_element(all(v)); }
template <typename T> T acc(const vector<T> &v) { return std::accumulate(v.begin(), v.end(), T(0LL)); }
template <typename T> istream& operator>>(istream& is, std::vector<T>& v) { for(auto& x : v) { is >> x; } return is; }

}
using namespace defs;
\end{lstlisting}

\subsection{Int128}
\begin{lstlisting}
#if defined(__GNUC__) || defined(__clang__)
using i128 = __int128;
using u128 = unsigned __int128;
#else
#error "int128 is only supported on GCC and Clang compilers"
#endif

namespace std {
template <>
class numeric_limits<i128> {
public:
    static constexpr bool is_specialized = true;
    static constexpr i128 min() { return static_cast<u128>(1) << 127; }
    static constexpr i128 max() { return ~(static_cast<u128>(1) << 127); };
};
} // namespace std

std::ostream& operator<<(std::ostream& os, i128 n) {
    if (n == 0) return os << '0';
    
    const bool is_negative = n < 0;
    u128 abs_n = is_negative ? -static_cast<u128>(n) : static_cast<u128>(n);
    
    char buffer[40] = {0};
    char* ptr = buffer + sizeof(buffer) - 1;
    
    while (abs_n > 0) {
        *--ptr = '0' + abs_n % 10;
        abs_n /= 10;
    }
    
    if (is_negative) *--ptr = '-';
    return os << ptr;
}


std::istream& operator>>(std::istream& is, i128& n) {
    std::string s;
    is >> s;
    
    try {
        n = toi128(s);
    } catch (const std::exception& e) {
        is.setstate(std::ios::failbit);
        throw;
    }
    return is;
}

i128 toi128(const std::string& s) {
    if (s.empty()) throw std::invalid_argument("Empty input string");
    
    size_t pos = 0;
    const bool negative = (s[0] == '-');
    if (negative || s[0] == '+') pos++;
    if (pos >= s.size()) throw std::invalid_argument("Invalid number format");
    
    constexpr i128 max_prev = std::numeric_limits<i128>::max() / 10;
    constexpr i128 max_digit = std::numeric_limits<i128>::max() % 10;
    
    i128 result = 0;
    for (; pos < s.size(); ++pos) {
        if (!std::isdigit(s[pos])) {
            throw std::invalid_argument("Non-digit character in input");
        }
        
        const int digit = s[pos] - '0';
        if (result > max_prev || (result == max_prev && digit > max_digit + negative)) {
            throw std::overflow_error("i128 overflow");
        }
        
        result = result * 10 + digit;
    }
    return negative ? -result : result;
}

i128 sqrti128(i128 n) {
    if (n < 0) throw std::domain_error("Square root of negative number");
    if (n == 0) return 0;
    
    i128 low = 0;
    i128 high = (static_cast<u128>(1) << 63) - 1;
    
    while (low < high) {
        const i128 mid = (low + high + 1) / 2;
        
        bool overflow = false;
        i128 square;
        if (mid > std::numeric_limits<i128>::max() / mid) {
            overflow = true;
        } else {
            square = mid * mid;
        }
        
        if (overflow || square > n) {
            high = mid - 1;
        } else {
            low = mid;
        }
    }
    return low;
}

i128 gcd(i128 a, i128 b) noexcept {
    if (a == 0) return b;
    if (b == 0) return a;
    
    int shift = 0;
    while (((a | b) & 1) == 0) {
        a >>= 1;
        b >>= 1;
        ++shift;
    }
    
    while (a != b) {
        while ((a & 1) == 0) a >>= 1;
        while ((b & 1) == 0) b >>= 1;
        if (a > b) std::swap(a, b);
        b -= a;
    }
    return a << shift;
}
\end{lstlisting}

\subsection{二分搜索}
\begin{lstlisting}
// return the first ans in [lo, hi], such as check(md) = true
// if no such ans, return hi + 1
template<class T, class Func>
T binary_min_left(T lo, T hi, Func check) {
	  T ans = hi + 1;
	  while(lo <= hi) {
		    T md = lo + (hi - lo) >> 1;
        if(check(md)) {
            ans = md;
            hi = md - 1;
        } else {
            lo = md + 1;
        }
    }
    return ans;
}


// return the last ans in [lo, hi], such as check(md) = true
// if no such ans, return lo - 1
template<class T, class Func>
T binary_max_right(T lo, T hi, Func check) {
    T ans = lo - 1;
    while(lo <= hi) {
        T md = lo + (hi - lo) >> 1;
        if(check(md)) {
            ans = md;
            lo = md + 1;
        } else {
            hi = md - 1;
        }
    }
	  return ans;
}
\end{lstlisting}

\subsection{自定义哈希}
\begin{lstlisting}
struct custom_hash_64 {
    static uint64_t splitmix64(uint64_t x) {
        x += 0x9e3779b97f4a7c15;
        x = (x ^ (x >> 30)) * 0xbf58476d1ce4e5b9;
        x = (x ^ (x >> 27)) * 0x94d049bb133111eb;
        return x ^ (x >> 31);
    }

    size_t operator()(uint64_t x) const {
        static const uint64_t FIXED_RANDOM = 
            std::chrono::steady_clock::now().time_since_epoch().count();
        return splitmix64(x ^ FIXED_RANDOM);
    }
};

struct custom_hash_32 {
    uint64_t operator()(uint32_t x) const {
        static const uint32_t RANDOM = 
            std::chrono::steady_clock::now().time_since_epoch().count();
        return (x ^ RANDOM) * 0x9e3779b1;
    }
};

// unordered_map<int, int, custom_hash_xxx> cnt;
\end{lstlisting}

\subsection{对拍}
\begin{lstlisting}
#include <iostream>
using namespace std;
int main() {
    // 编译data文件
    system("g++ data.cpp -o data.exe --std=c++11");
    // 编译 std文件
    system("g++ std.cpp -o std.exe --std=c++11");
    // 编译force文件
    system("g++ force.cpp -o force.exe --std=c++11"); 
    int rp = 0;
    while (!rp) {
        // 生成数据,将结果存储在 input.txt中
        system("data.exe > input.txt");
        // 调用暴力代码,将结果存储在 fout.txt 中
        system("force.exe < input.txt > fout.txt");
        // 调用对比代码,将结果存储在 sout.txt 中
        system("std.exe < input.txt > sout.txt");
        rp = system("fc fout.txt sout.txt"); // 进行对比
        if (rp == 0)  {
            cout << "AC" << endl;
        } else {
            cout << "WA" << endl;
            break;
        }
    }
    return 0;
}
\end{lstlisting}