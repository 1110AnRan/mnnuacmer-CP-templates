\section{平面几何专题模板}
\subsection{平面几何}
\begin{lstlisting}
template<class T>
int sgn(const T& v) {
    static constexpr T eps = 1e-8;
    return v > eps ? 1 : v < -eps ? -1 : 0;
}
template<class T>
struct Point {// Point or Vector
    T x, y;
    Point() : x(0), y(0) {}
    Point(T x, T y) : x(x), y(y) {}
    template<class U>
    explicit operator Point<U>() const {
        return Point<U>(U(x), U(y));
    }
    Point operator+(const Point& o) const {
        return Point(x + o.x, y + o.y);
    }
    Point operator-(const Point& o) const {
        return Point(x - o.x, y - o.y);
    }
    Point operator-() const {
        return Point(-x, -y);
    }
    Point operator*(const T& v) const {
        return Point(x * v, y * v);
    }
    friend Point operator*(const T& v, const Point<T>& o) {
        return Point(o.x * v, o.y * v);
    }
    Point operator/(const T& v) const {
        return Point(x / v, y / v);
    }
    Point operator+=(const Point& o) {
        x += o.x, y += o.y;
        return *this;
    }
    Point operator-=(const Point& o) {
        x -= o.x, y -= o.y;
        return *this;
    }
    Point operator*=(const T& v) {
        x *= v, y *= v;
        return *this;
    }
    Point operator/=(const T& v) {
        x /= v, y /= v;
        return *this;
    }
    bool operator==(const Point& o) const {
        return sgn(x - o.x) == 0 and sgn(y - o.y) == 0;
    }
    bool operator!=(const Point& o) const {
        return sgn(x - o.x) != 0 or sgn(y - o.y) != 0;
    }
    bool operator<(const Point& o) const {
        return sgn(x - o.x) < 0 or sgn(x - o.x) == 0 and sgn(y - o.y) < 0;
    }
    bool operator>(const Point& o) const {
        return sgn(x - o.x) > 0 or sgn(x - o.x) == 0 and sgn(y - o.y) > 0;
    }
    static bool argcmp(const Point& a, const Point& b) {
        static auto get = [&](const Point& o) {
            if(sgn(o.x) == 0 and sgn(o.y) == 0) return 0;
            if(sgn(o.y) > 0 or sgn(o.y) == 0 and sgn(o.x) < 0) return 1;
            return -1;
        };
        int ta = get(a), tb = get(b);
        if(ta != tb) return ta < tb;
        return a.toLeft(b) == 1;// 不关注极径
        // int tole = a.toLeft(b);
        // if(tole != 0) return tole == 1;
        // return sgn(a.square()-b.square()) < 0;// 极角相同按极径排
    }
    T dot(const Point& o) const {
        return x * o.x + y * o.y;
    }
    T cross(const Point& o) const {
        return x * o.y - y * o.x;
    }
    int toLeft(const Point& o) const {
        return sgn(cross(o));
    }
    T square() const {
        return x * x + y * y;
    }
    T interSquare(const Point& o) const {
        return (*this - o).square();
    }
    friend istream& operator>>(istream& in, Point& o) {
        return in >> o.x >> o.y;
    }
    friend ostream& operator<<(ostream& out, Point const& o) {
        return out << "(" << o.x << "," << o.y << ")";
    }
    // 涉及浮点数
    double length() const {
        return sqrtl(square());
    }
    double distance(const Point& o) const {
        return (*this - o).length();
    }
    // 逆时针旋转 rad
    template<class U>
    Point<U> rotate(U cosr, U sinr) const {
        return Point(x * cosr - y * sinr, x * sinr + y * cosr);
    }
    // 两向量夹角范围是 [0,PI]
    double ang(const Point& o) const {
        return acosl(max(-1.0l, min(1.0l, dot(o) / (length() * o.length()))));
    }
};
template<class T>
struct Line {// Line or Segment
    Point<T> a, b;// 方向为 a->b
    Line() {}
    Line(const Point<T>& a, const Point<T>& b) : a(a), b(b) {}
    template<class U>
    Line(const Point<U>& a, const Point<U>& b) : a(a), b(b) {}
    Point<T> vec() const {
        return b - a;
    }
    // Line ----------------------------------------------------------
    bool parallel(const Line& l) const {
        return sgn((b - a).cross(l.b - l.a)) == 0;
    }
    int toLeft(const Point<T>& o) const {
        return (b - a).toLeft(o - a);
    }
    // 涉及浮点数
    // 直线交点
    Point<double> lineIntersection(const Line& l) const {
        return Point<double>(a) + Point<double>(b - a) *
            (1. * (l.b - l.a).cross(a - l.a) / (l.b - l.a).cross(a - b));
    }
    // 点到直线的距离
    double distanceLP(const Point<T>& o) const {
        return abs((a - b).cross(a - o)) / (a - b).length();
    }
    // 点在直线上的投影
    Point<T> projection(const Point<T>& o) const {
        return a + (b - a) * (1. * (b - a).dot(o - a) / (b - a).square());
    }
    // Segment ----------------------------------------------------------
    // -1 点在线段端点 | 0 点不在线段上 | 1 点严格在线段上
    int contain(const Point<T>& o) const {
        if(o == a or o == b) return -1;
        return (o - a).toLeft(o - b) == 0 and sgn((o - a).dot(o - b)) < 0;
    }
    // 判断线段直线是否相交
    // 0 线段和直线不相交 | 1 线段和直线严格相交 | 2 仅在某一线段端点处相交 | 3 直线包含线段
    int interWithLine(const Line& l) const {
        int num = !l.toLeft(a) + !l.toLeft(b);
        if(num) return num + 1;
        return l.toLeft(a) != l.toLeft(b);
    }
    // 判断两线段是否相交
    // 0 两线段不相交 | 1 两线段严格相交 | 2 仅在某一线段端点处相交 | 3 两线段有重叠
    int interWithSegment(Line s) const {
        if((a < b) != (s.a < s.b))
            swap(s.a, s.b);
        int num = (contain(s.a) != 0) + (contain(s.b) != 0)
            + (s.contain(a) != 0) + (s.contain(b) != 0);
        if(parallel(s)) {
            if(!num) return 0;
            if(b == s.a or a == s.b) return 2;// -.-
            return 3;
        }
        if(num) return 2;
        return toLeft(s.a) * toLeft(s.b) == -1 and s.toLeft(a) * s.toLeft(b) == -1;
    }
    // 点到线段的距离
    double distanceSP(const Point<T>& o) const {
        if(sgn((o - a).dot(b - a)) < 0) return o.distance(a);
        if(sgn((o - b).dot(a - b)) < 0) return o.distance(b);
        return abs((a - b).cross(a - o)) / (a - b).length();
    }
    // 两线段间距离
    double distanceSS(const Line& s) const {
        if(interWithSegment(s)) return 0;
        return min({distanceSP(s.a),distanceSP(s.b),
                s.distanceSP(a),s.distanceSP(b)});
    }
};
template<class T>
struct Polygon {
    int n;
    vector<Point<T>> p;
    // p 以逆时针顺序存储 2 遍
    Polygon(vector<Point<T>> const& p_) : n(p_.size()), p(p_) {
        p.insert(p.end(), p_.begin(), p_.end());
    }
    // 返回 回转数 = 逆时针转头圈数-顺时针转头圈数
    // 1e9 在多边形上 | 0 不在多边形内 | !=0 在多边形内
    int contain(const Point<T>& o) const {
        int cnt = 0;
        for(int i = 0; i < n; ++i) {
            Point<T> const& u = p[i], v = p[i + 1];
            Line<T> const l(u, v);
            if(l.contain(o)) return 1e9;
            cnt += l.toLeft(o) > 0 and sgn(u.y - o.y) < 0 and sgn(v.y - o.y) >= 0;
            cnt -= l.toLeft(o) < 0 and sgn(u.y - o.y) >= 0 and sgn(v.y - o.y) < 0;
        }
        return cnt;
    }
    // 多边形面积的两倍,可用于判断点的存储顺序是顺时针或逆时针(逆正顺负)
    T area() const {
        T sum = 0;
        for(int i = 0; i < n; ++i)
            sum += p[i].cross(p[i + 1]);
        return sum;
    }
    // 多边形的周长
    double perimeter() const {
        double sum = 0;
        for(int i = 0; i < n; ++i)
            sum += p[i].distance(p[i + 1]);
        return sum;
    }
};

template<class T>
struct Convex : Polygon<T> {
    using Polygon<T>::n;
    using Polygon<T>::p;
    Convex(vector<Point<T>> const& p, bool keepRaw) : Polygon<T>(p) {}
    Convex(vector<Point<T>> const& p_) : Polygon<T>(andrew(p_)) {}
    // 对点集 p 求凸包
    static auto andrew(vector<Point<T>> p) {
        sort(p.begin(), p.end());
        p.erase(unique(p.begin(), p.end()), p.end());
        if(p.size() <= 1) return p;
        vector<Point<T>> st;
        for(auto& e : p) {
            while(st.size() > 1 and
                (st.back() - st.end()[-2]).toLeft(e - st.back()) <= 0)
                st.pop_back();
            st.push_back(e);
        }
        int sz = st.size();
        for(int i = (int)p.size() - 2; i >= 0; --i) {
            while(st.size() > sz and
                (st.back() - st.end()[-2]).toLeft(p[i] - st.back()) <= 0)
                st.pop_back();
            st.push_back(p[i]);
        }
        st.pop_back();
        return st;
    }
    // O(logn)判断点是否在凸多边形内
    // -1 在边界上 | 0 在外部 | 1 严格在内部
    int contain(const Point<T>& o) const {
        if(n == 1) return p[0] == o ? -1 : 0;
        int fTo = (p[1] - p[0]).toLeft(o - p[0]);
        int bTo = (p.back() - p[0]).toLeft(o - p[0]);
        if(fTo == -1 or bTo == 1) return 0;
        if(fTo == 0) return sgn((o - p[0]).dot(o - p[1])) <= 0 ? -1 : 0;
        if(bTo == 0) return sgn((o - p[0]).dot(o - p.back())) <= 0 ? -1 : 0;

        int i = partition_point(p.begin() + 2, p.begin() + n, [&](Point<T> const& v) {
            return (v - p[0]).toLeft(o - p[0]) >= 0;
        }) - p.begin();
        Line<T> const l(p[i - 1], p[i]);
        return l.contain(o) ? -1 : l.toLeft(o) == 1;
    }
    // O(logn) 二分找到 f 方向上的切点 i,满足 p[i]-p[i-1],f(p[i]),p[i+1]-p[i] 逆时旋转
    template<class Func>
    int extreme(Func const& f) {
        assert(n > 2);
        Point<ll> const divVec = f(p[0]);
        bool const flag = (p[0] - p[n - 1]).toLeft(divVec) < 0;
        return partition_point(p.begin(), p.begin() + n, [&](Point<T> const& a) {
            if(divVec.toLeft(a - p[0]) > 0) return flag;
            return (*(&a + 1) - a).toLeft(f(a)) > 0;
        }) - p.begin();
    }
    // O(logn) 二分找到 v 方向 和 -v 方向上的切点,返回值切点下标 in [0,n-1]
    array<int, 2> tangentByLine(Point<T> const& v) {
        int i = extreme([&](...) {return v; });
        int j = extreme([&](...) {return -v; });
        return {i,j};
    }
    // O(logn) 过点 o 向凸包做两条切线(先左后右),返回值切点下标 in [0,n-1]
    // 需要保证 o 在凸包外面
    array<int, 2> tangentByPoint(Point<T> const& o) {
        int i = extreme([&](Point<T> const& a) {return o - a; });
        int j = extreme([&](Point<T> const& a) {return a - o; });
        return {i, j};
    }
};
\end{lstlisting}

\subsection{线段在多边形内}
\begin{lstlisting}
// struct Polygon {
// O(|p|) 判断线段在多边形内
// 可以用整型判断 / 如果用浮点型,可能要把精度调松一点
bool contain(const Line<T>& s) const {
    if(!contain(s.a) or !contain(s.b))
        return false;
    if(s.a == s.b)
        return true;
    vector<int> t(p.size());
    for(int i=0; i<p.size(); ++i) {
        auto& u = p[i];
        auto& v = p[nxt(i)];
        Line<T> uv(u,v);
        t[i] = s.interWithSegment(uv);
        if(t[i] == 0) continue;// not intersect
        if(t[i] == 1) return false;// strickly intersect
        if(t[i] == 2 and uv.contain(s.a)==1 and uv.toLeft(s.b)==-1)
            return false;
        if(t[i] == 3) {// overlap
            if(s.contain(v)==1 and uv.toLeft(p[nxt(nxt(i))])==1)
                return false;
            if(s.contain(u)==1 and uv.toLeft(p[pre(i)])==1)
                return false;
        }
    }
    for(int i=0; i<p.size(); ++i) {
        if(!(t[i]==2 and t[nxt(i)]==2)) continue;
        auto& v = p[nxt(i)];
        // intersect at v
        if(s.contain(v) and s.b != v) {
            auto& u = p[i];
            auto& w = p[nxt(nxt(i))];
            if((v-u).toLeft(w-u)==1) {
                if(s.toLeft(u)==1 and s.toLeft(w)==-1);
                else return false;
            }else {
                if(s.toLeft(u)==-1 and s.toLeft(w)==1)
                    return false;
            }
        }
    }
    return true;
}
\end{lstlisting}

\subsection{钝角直角三角形计数问题}
\begin{lstlisting}
for(int k=0; k<n; ++k)  {
    vector<Point<ll>> b;
    b.reserve(n-1);
    for(int i=0; i<n; ++i) {
        if(a[i] == a[k]) continue;
        b.emplace_back(a[i]-a[k]);
    }
    sort(ALL(b), Point<ll>::argcmp);
    int sz = b.size();
    if(!sz) continue;
    b.insert(b.end(), b.begin(), b.end());
    for(int i=0,l=0,r=0; i<sz; ++i) {
        // 0
        auto eq0 = [&](int j) {
            return b[i].cross(b[j])==0 and b[i].dot(b[j])>0;
        };
        // [0,89]
        auto le89 = [&](int j) {
            return b[i].cross(b[j])>=0 and b[i].dot(b[j])>0;
        };
        // [0,180)
        auto le179 = [&](int j) {
            int t = b[i].toLeft(b[j]);
            return t>0 or t==0 and b[i].dot(b[j])==0;
        };
        // [l,r) -> [90,180)
        l = max(l, i);
        while(l<i+sz and le89(l)) l++;
        r = max(r, l);
        while(r<i+sz and le179(r) and !eq0(r)) r++;
        ans += r-l;
    }
}
\end{lstlisting}

\subsection{向量夹角}
\begin{lstlisting}
double alpha = atan2(v.cross(w), v.dot(w));// v,w 不是零向量
\end{lstlisting}

\subsection{凸包上旋转卡尺算法的其他应用}
\begin{lstlisting}
// struct Convex {
// 旋转卡尺求直径的平方
T rotatingCalipers() const {
    if(p.size()==1) return 0;
    if(p.size()==2) return p[0].interSquare(p[1]);
    T ans = 0;
    for(int i=0, j=1; i<p.size(); ++i) {
        Point<T> v = p[nxt(i)]-p[i];
        while(v.cross(p[nxt(j)]-p[i]) > v.cross(p[j]-p[i]))
            j=nxt(j);
        ans = max({ans, p[i].interSquare(p[j]), 
                    p[nxt(i)].interSquare(p[j])});
        if(v.cross(p[nxt(j)]-p[i]) == v.cross(p[j]-p[i]))
            ans = max({ans, p[i].interSquare(p[nxt(j)]), 
                        p[nxt(i)].interSquare(p[nxt(j)])});
    }
    return ans;
}
// 结论:覆盖凸包的最小面积/周长矩形,一定有一条边和凸包某条边重叠
// 旋转卡尺求最小面积矩形
double minErea() const {
    if(p.size()<=2) return 0;
    double ans = numeric_limits<double>::max();
    for(int i=0, j=1, k=1, l=1; i<p.size(); ++i) {
        Point<T> v = p[nxt(i)]-p[i];
        while(v.cross(p[nxt(j)]-p[i]) > v.cross(p[j]-p[i])) j = nxt(j);
        while(v.dot(p[nxt(k)]-p[i]) > v.dot(p[k]-p[i])) k = nxt(k);
        if(!i) l = j;
        while(v.dot(p[nxt(l)]-p[i]) < v.dot(p[l]-p[i])) l = nxt(l);
        ans = min(ans, 1. * v.cross(p[j]-p[i]) / v.square()
                        * (v.dot(p[k]-p[i]) - v.dot(p[l]-p[i])));
    }
    return ans;
}
// 旋转卡尺求最小宽度
double minWidth() const {
    if(p.size()<=2) return 0;
    double ans = numeric_limits<double>::max();
    for(int i=0, j=1; i<p.size(); ++i) {
        Point<T> v = p[nxt(i)]-p[i];
        while(v.cross(p[nxt(j)]-p[i]) > v.cross(p[j]-p[i])) j = nxt(j);
        ans = min(ans, v.cross(p[j]-p[i])/v.length());
    }
    return ans;
}
// 计算两个相离的凸包之间的最短距离,注意调用两次取min
double distance(const Convex<double>& B) {
    double ans = numeric_limits<double>::max();
    for(int i=0,j=0; i<p.size(); ++i) {
        Point<double> v(p[nxt(i)]-p[i]);
        Line<double> s(p[i], p[nxt(i)]);
        if(i == 0) {
            double mx = numeric_limits<double>::min();
            for(int k=0; k<B.p.size(); ++k) {
                double cro = v.cross(B.p[k]-p[i]);
                if(sgn(cro-mx)>0) {
                    mx = cro;
                    j = k;
                }
            }
            ans = min(ans, s.distanceSP(!j ? B.p.back(): B.p[j-1]));
        }
        ans = min(ans, s.distanceSP(B.p[j]));
        while(sgn(v.cross(B.p[B.nxt(j)]-p[i])-v.cross(B.p[j]-p[i])) >= 0) {
            j = B.nxt(j);
            ans = min(ans, s.distanceSP(B.p[j]));
        }
    }
    return ans;
}
\end{lstlisting}

\subsection{最大最小三角形}
\begin{lstlisting}
// 求出的是最大/最小三角形面积的两倍
template<class T>
pair<T,T> minMaxTriangle(const vector<Point<T>>& a) {
    int n = a.size();
    T mn = numeric_limits<T>::max();
    T mx = 0;
    using Node = tuple<int,int,Point<T>>;
    vector<Node> all;
    all.reserve(n*n);
    vector<int> id(n), pos(n);
    for(int i=0; i<n; ++i) {
        id[i] = i;
        for(int j=0; j<n; ++j) {
            if(i==j) continue;
            if(a[i]==a[j]) mn=0;
            else all.emplace_back(i,j,a[j]-a[i]);
        }
    }
    sort(all.begin(), all.end(), [&](const Node& x,const Node& y) {
        return Point<T>::argcmp(get<2>(x), get<2>(y));
    });
    sort(id.begin(), id.end(), [&](const int& i,const int& j) {
        return a[i].y < a[j].y or a[i].y==a[j].y and a[i].x>a[j].x;
    });
    for(int i=0; i<n; ++i)
        pos[id[i]] = i;
    for(auto [i,j,v]:all) {
        // 如果没有三点共线 assert(pos[i] = pos[j] + 1);
        if(pos[i] > pos[j]) {
            swap(id[pos[i]], id[pos[j]]);
            swap(pos[i], pos[j]);
        }
        int t = max(pos[i],pos[j])+1;
        if(t<n) {
            mn = min(mn, (a[id[t]]-a[i]).cross(v));
            mx = max(mx, (a[id.back()]-a[i]).cross(v));
        }
    }
    return {mn, mx};
}
\end{lstlisting}

\subsection{动态凸包}
\begin{lstlisting}
template<class T>
struct DynamicConvex {
    /// @note operator< 使用极角序,并考虑极径
    Point<T> o;
    set<Point<T>> s; // 坐标扩大三倍,使得三角形中心为整数
    using Iter = decltype(s.begin());
    auto nxt(Iter it) const {
        return next(it) == s.end() ? s.begin() : next(it);
    }
    auto pre(Iter it) const {
        return it == s.begin() ? --s.end() : prev(it);
    }
    bool contain(Point<T> const& a) const {
        if(s.size() == 0) return 0;
        if(s.size() == 1) return *s.begin() == a * 3;
        if(s.size() == 2) return Line<T>(*s.begin(), *s.rbegin()).contain(a);
        auto it = s.lower_bound(a * 3 - o);
        if(it == s.end()) it = s.begin();
        return (*it - *pre(it)).toLeft(a * 3 - o - *pre(it)) >= 0;
    }
    void add(Point<T> a) {
        if(s.size() <= 1) {
            s.insert(a * 3);
            return;
        }
        if(s.size() == 2) {
            auto u = *s.begin(), v = *s.rbegin();
            if((u - v).toLeft(a * 3 - v) == 0) return;
            o = (u + v + a * 3) / 3;
            s = {u - o, v - o, a * 3 - o};
            for(auto it = s.begin(); it != s.end(); ++it) addEdge(it, nxt(it));
            return;
        }
        if(contain(a)) return;
        a = a * 3 - o;
        auto it = s.insert(a).first, np = nxt(it), pp = pre(it);
        delEdge(pp, np);
        while(s.size() > 3 and ((*np - a).toLeft(*nxt(np) - *np)) != 1) {
            delEdge(np, nxt(np));
            s.erase(np);
            np = nxt(it);
        }
        while(s.size() > 3 and ((*pp - *pre(pp)).toLeft(a - *pp)) != 1) {
            delEdge(pre(pp), pp);
            s.erase(pp);
            pp = pre(it);
        }
        addEdge(pre(it), it);
        addEdge(it, nxt(it));
    }

    double D;// 周长
    map<Point<T>, Iter> edge;
    void addEdge(Iter it, Iter nit) {// s.size()>=3 时维护信息
        D += it->distance(*nit);
        edge[*nit - *it] = it;
    }
    void delEdge(Iter it, Iter nit) {
        D -= it->distance(*nit);
        edge.erase(*nit - *it);
    }
    /// @note 调用前注意直线 ax+by=c 的坐标扩大三倍 ax+by=3c
    array<Point<T>, 2> extremeByLine(Point<T> const& v) const {
        assert(s.size() > 2);
        auto get = [&](Point<T> const& v) {
            auto it = edge.lower_bound(v);
            if(it == edge.end()) it = edge.begin();
            return *(it->second) + this->o;
        };
        return {get(v), get(-v)};
    }
};
\end{lstlisting}

\subsection{闵可夫斯基和}
\begin{lstlisting}
// A+B = {a+b | a \in A, b \in B},复杂度 O(n)
template<class T>
Convex<T> MinkowskiSum(Convex<T> const& A, Convex<T> const& B) {
    auto cmp = [&](Point<T> const& a, Point<T> const& b) {
        return a.y > b.y or a.y == b.y and a.x < b.x;
    };
    int a = min_element(A.p.begin(), A.p.begin() + A.n, cmp) - A.p.begin();
    int b = min_element(B.p.begin(), B.p.begin() + B.n, cmp) - B.p.begin();
    Point<T> s(A.p[a] + B.p[b]);
    vector<Point<T>> ps(1, s);
    auto popC = [&](Point<T> const& e, Point<T> const& f) {
        return (e - f).toLeft(s - e) == 0 and sgn((e - f).dot(s - e)) >= 0;
    };
    auto f = [&](int owner, int i) {
        return !owner ? A.p[a + i + 1] - A.p[a + i] : B.p[b + i + 1] - B.p[b + i];
    };
    for(int i = 0, j = 0; i < A.n or j < B.n; ) {
        if(j >= B.n or i < A.n and Point<T>::argcmp(f(0, i), f(1, j))) s += f(0, i++);
        else s += f(1, j++);
        while(ps.size() > 1 and popC(ps.back(), ps.end()[-2])) ps.pop_back();
        ps.emplace_back(s);
    }
    ps.pop_back();
    return Convex<T>(ps, true);
};
\end{lstlisting}

\subsection{半平面交}
\begin{lstlisting}
template<class T>
vector<Line<T>> hp(vector<Line<T>> vs, T inf = T(1e9)) {
    vs.emplace_back(Point<T>(inf, -inf), Point<T>(inf, inf));
    vs.emplace_back(Point<T>(inf, inf), Point<T>(-inf, inf));
    vs.emplace_back(Point<T>(-inf, inf), Point<T>(-inf, -inf));
    vs.emplace_back(Point<T>(-inf, -inf), Point<T>(inf, -inf));
    auto sameDir = [&](Line<T> const& a, Line<T> const& b) {
        return a.parallel(b) and sgn(a.vec().dot(b.vec())) >= 0;
    };
    sort(vs.begin(), vs.end(), [&](Line<T> const& a, Line<T> const& b) {
        if(sameDir(a, b)) return a.toLeft(b.a) == -1;
        return Point<T>::argcmp(a.vec(), b.vec());
    });
    auto canPop = [&](Line<T> const& a, Line<T> const& b, Line<T> const& c) {
        if constexpr(!is_same_v<T, double>) {
            __int128_t x = (c.b - c.a).cross(b.a - c.a), y = (c.b - c.a).cross(b.a - b.b);
            using P = Point<__int128_t>;
            return P(a.vec()).toLeft(P(b.a) * y + P(b.vec()) * x - P(a.a) * y) == -sgn(y);
        }
        return Point<double>(a.vec()).toLeft(b.lineIntersection(c) - Point<double>(a.a)) < 0;
    };
    deque<Line<T>> q;
    for(auto& v : vs) {
        if(q.size() and sameDir(q.back(), v)) continue;
        while(q.size() > 1 and canPop(v, q.back(), q[q.size() - 2])) q.pop_back();
        while(q.size() > 1 and canPop(v, q[0], q[1])) q.pop_front();
        if(q.size() and q.back().vec().toLeft(v.vec()) <= 0) return {};
        q.push_back(v);
    }
    while(q.size() > 1 and canPop(q[0], q.back(), q[q.size() - 2])) q.pop_back();
    while(q.size() > 1 and canPop(q.back(), q[0], q[1])) q.pop_front();
    return vector<Line<T>>(q.begin(), q.end());
}
\end{lstlisting}
