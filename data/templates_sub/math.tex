\section{数学专题模板}

\subsection{Exgcd}
\begin{lstlisting}
int exgcd(int a, int b, int &x, int &y) {
	if (b == 0) {
		x = 1, y = 0;
		return a;
	}
	int d = exgcd(b, a % b, y, x);
	y -= (a / b * x);
	return d;
}
\end{lstlisting}


\subsection{Frac}
\begin{lstlisting}
template<class T>
struct Frac {
    T num;
    T den;
    Frac(T num_, T den_) : num(num_), den(den_) {
        if (den < 0) {
            den = -den;
            num = -num;
        }
    }
    Frac() : Frac(0, 1) {}
    Frac(T num_) : Frac(num_, 1) {}
    explicit operator double() const {
        return 1. * num / den;
    }
    Frac &operator+=(const Frac &rhs) {
        num = num * rhs.den + rhs.num * den;
        den *= rhs.den;
        return *this;
    }
    Frac &operator-=(const Frac &rhs) {
        num = num * rhs.den - rhs.num * den;
        den *= rhs.den;
        return *this;
    }
    Frac &operator*=(const Frac &rhs) {
        num *= rhs.num;
        den *= rhs.den;
        return *this;
    }
    Frac &operator/=(const Frac &rhs) {
        num *= rhs.den;
        den *= rhs.num;
        if (den < 0) {
            num = -num;
            den = -den;
        }
        return *this;
    }
    friend Frac operator+(Frac lhs, const Frac &rhs) {
        return lhs += rhs;
    }
    friend Frac operator-(Frac lhs, const Frac &rhs) {
        return lhs -= rhs;
    }
    friend Frac operator*(Frac lhs, const Frac &rhs) {
        return lhs *= rhs;
    }
    friend Frac operator/(Frac lhs, const Frac &rhs) {
        return lhs /= rhs;
    }
    friend Frac operator-(const Frac &a) {
        return Frac(-a.num, a.den);
    }
    friend bool operator==(const Frac &lhs, const Frac &rhs) {
        return lhs.num * rhs.den == rhs.num * lhs.den;
    }
    friend bool operator!=(const Frac &lhs, const Frac &rhs) {
        return lhs.num * rhs.den != rhs.num * lhs.den;
    }
    friend bool operator<(const Frac &lhs, const Frac &rhs) {
        return lhs.num * rhs.den < rhs.num * lhs.den;
    }
    friend bool operator>(const Frac &lhs, const Frac &rhs) {
        return lhs.num * rhs.den > rhs.num * lhs.den;
    }
    friend bool operator<=(const Frac &lhs, const Frac &rhs) {
        return lhs.num * rhs.den <= rhs.num * lhs.den;
    }
    friend bool operator>=(const Frac &lhs, const Frac &rhs) {
        return lhs.num * rhs.den >= rhs.num * lhs.den;
    }
    friend std::ostream &operator<<(std::ostream &os, Frac x) {
        T g = std::gcd(x.num, x.den);
        if (x.den == g) {
            return os << x.num / g;
        } else {
            return os << x.num / g << "/" << x.den / g;
        }
    }
};
\end{lstlisting}


\subsection{ModInt}
\begin{lstlisting}
template<class T>
constexpr T power(T a, u64 b, T res = 1) {
    for (; b != 0; b /= 2, a *= a) {
        if (b & 1) {
            res *= a;
        }
    }
    return res;
}
 
template<u32 P>
constexpr u32 mulMod(u32 a, u32 b) {
    return u64(a) * b % P;
}
 
template<u64 P>
constexpr u64 mulMod(u64 a, u64 b) {
    u64 res = a * b - u64(1.L * a * b / P - 0.5L) * P;
    res %= P;
    return res;
}
 
constexpr i64 safeMod(i64 x, i64 m) {
    x %= m;
    if (x < 0) {
        x += m;
    }
    return x;
}
 
constexpr std::pair<i64, i64> invGcd(i64 a, i64 b) {
    a = safeMod(a, b);
    if (a == 0) {
        return {b, 0};
    }
    
    i64 s = b, t = a;
    i64 m0 = 0, m1 = 1;
 
    while (t) {
        i64 u = s / t;
        s -= t * u;
        m0 -= m1 * u;
        
        std::swap(s, t);
        std::swap(m0, m1);
    }
    
    if (m0 < 0) {
        m0 += b / s;
    }
    
    return {s, m0};
}
 
template<std::unsigned_integral U, U P>
struct ModIntBase {
public:
    constexpr ModIntBase() : x(0) {}
    template<std::unsigned_integral T>
    constexpr ModIntBase(T x_) : x(x_ % mod()) {}
    template<std::signed_integral T>
    constexpr ModIntBase(T x_) {
        using S = std::make_signed_t<U>;
        S v = x_ % S(mod());
        if (v < 0) {
            v += mod();
        }
        x = v;
    }
    
    constexpr static U mod() {
        return P;
    }
    
    constexpr U val() const {
        return x;
    }
    
    constexpr ModIntBase operator-() const {
        ModIntBase res;
        res.x = (x == 0 ? 0 : mod() - x);
        return res;
    }
    
    constexpr ModIntBase inv() const {
        return power(*this, mod() - 2);
    }
    
    constexpr ModIntBase &operator*=(const ModIntBase &rhs) & {
        x = mulMod<mod()>(x, rhs.val());
        return *this;
    }
    constexpr ModIntBase &operator+=(const ModIntBase &rhs) & {
        x += rhs.val();
        if (x >= mod()) {
            x -= mod();
        }
        return *this;
    }
    constexpr ModIntBase &operator-=(const ModIntBase &rhs) & {
        x -= rhs.val();
        if (x >= mod()) {
            x += mod();
        }
        return *this;
    }
    constexpr ModIntBase &operator/=(const ModIntBase &rhs) & {
        return *this *= rhs.inv();
    }
    
    friend constexpr ModIntBase operator*(ModIntBase lhs, const ModIntBase &rhs) {
        lhs *= rhs;
        return lhs;
    }
    friend constexpr ModIntBase operator+(ModIntBase lhs, const ModIntBase &rhs) {
        lhs += rhs;
        return lhs;
    }
    friend constexpr ModIntBase operator-(ModIntBase lhs, const ModIntBase &rhs) {
        lhs -= rhs;
        return lhs;
    }
    friend constexpr ModIntBase operator/(ModIntBase lhs, const ModIntBase &rhs) {
        lhs /= rhs;
        return lhs;
    }
    
    friend constexpr std::istream &operator>>(std::istream &is, ModIntBase &a) {
        i64 i;
        is >> i;
        a = i;
        return is;
    }
    friend constexpr std::ostream &operator<<(std::ostream &os, const ModIntBase &a) {
        return os << a.val();
    }
    
    friend constexpr std::strong_ordering operator<=>(ModIntBase lhs, ModIntBase rhs) {
        return lhs.val() <=> rhs.val();
    }
    
private:
    U x;
};
 
template<u32 P>
using ModInt = ModIntBase<u32, P>;
template<u64 P>
using ModInt64 = ModIntBase<u64, P>;
 
struct Barrett {
public:
    Barrett(u32 m_) : m(m_), im((u64)(-1) / m_ + 1) {}
 
    constexpr u32 mod() const {
        return m;
    }
 
    constexpr u32 mul(u32 a, u32 b) const {
        u64 z = a;
        z *= b;
        
        u64 x = u64((u128(z) * im) >> 64);
        
        u32 v = u32(z - x * m);
        if (m <= v) {
            v += m;
        }
        return v;
    }
 
private:
    u32 m;
    u64 im;
};
 
template<u32 Id>
struct DynModInt {
public:
    constexpr DynModInt() : x(0) {}
    template<std::unsigned_integral T>
    constexpr DynModInt(T x_) : x(x_ % mod()) {}
    template<std::signed_integral T>
    constexpr DynModInt(T x_) {
        int v = x_ % int(mod());
        if (v < 0) {
            v += mod();
        }
        x = v;
    }
    
    constexpr static void setMod(u32 m) {
        bt = m;
    }
    
    static u32 mod() {
        return bt.mod();
    }
    
    constexpr u32 val() const {
        return x;
    }
    
    constexpr DynModInt operator-() const {
        DynModInt res;
        res.x = (x == 0 ? 0 : mod() - x);
        return res;
    }
    
    constexpr DynModInt inv() const {
        auto v = invGcd(x, mod());
        assert(v.first == 1);
        return v.second;
    }
    
    constexpr DynModInt &operator*=(const DynModInt &rhs) & {
        x = bt.mul(x, rhs.val());
        return *this;
    }
    constexpr DynModInt &operator+=(const DynModInt &rhs) & {
        x += rhs.val();
        if (x >= mod()) {
            x -= mod();
        }
        return *this;
    }
    constexpr DynModInt &operator-=(const DynModInt &rhs) & {
        x -= rhs.val();
        if (x >= mod()) {
            x += mod();
        }
        return *this;
    }
    constexpr DynModInt &operator/=(const DynModInt &rhs) & {
        return *this *= rhs.inv();
    }
    
    friend constexpr DynModInt operator*(DynModInt lhs, const DynModInt &rhs) {
        lhs *= rhs;
        return lhs;
    }
    friend constexpr DynModInt operator+(DynModInt lhs, const DynModInt &rhs) {
        lhs += rhs;
        return lhs;
    }
    friend constexpr DynModInt operator-(DynModInt lhs, const DynModInt &rhs) {
        lhs -= rhs;
        return lhs;
    }
    friend constexpr DynModInt operator/(DynModInt lhs, const DynModInt &rhs) {
        lhs /= rhs;
        return lhs;
    }
    
    friend constexpr std::istream &operator>>(std::istream &is, DynModInt &a) {
        i64 i;
        is >> i;
        a = i;
        return is;
    }
    friend constexpr std::ostream &operator<<(std::ostream &os, const DynModInt &a) {
        return os << a.val();
    }
    
    friend constexpr std::strong_ordering operator<=>(DynModInt lhs, DynModInt rhs) {
        return lhs.val() <=> rhs.val();
    }
    
private:
    u32 x;
    static Barrett bt;
};

constexpr int MOD = 1'000'000'007;
// constexpr int MOD = 998'244'353;
 
template<u32 Id>
Barrett DynModInt<Id>::bt = MOD;
 
using Z = ModInt<MOD>;
\end{lstlisting}


\subsection{Sieve}
\begin{lstlisting}
// 输入:n - 筛的范围上限
// 输出:
//   primes       - 质数集合
//   min_prime    - 每个数的最小质因子
//   phi         - 欧拉函数值
//   mu          - 莫比乌斯函数值
//   d           - 约数个数
//   cnt_min_p   - 最小质因子的次数
std::vector<int> primes, min_prime, phi, mu, d, cnt_min_p;
std::vector<bool> is_prime;
void euler_sieve(int n) {
    min_prime.resize(n + 1, 0);
    phi.resize(n + 1, 0);
    mu.resize(n + 1, 0);
    d.resize(n + 1, 0);
    cnt_min_p.resize(n + 1, 0);
    is_prime.resize(n + 1, true);
    phi[1] = mu[1] = d[1] = 1;
    for(int i = 2; i <= n; i++) {
        if(is_prime[i]) {
            primes.push_back(i);
            min_prime[i] = i;
            phi[i] = i - 1;
            mu[i] = -1;
            d[i] = 2;
            cnt_min_p[i] = 1;
        }
        for(auto p : primes) {
            int num = i * p;
            if(num > n) {
                break;
            }
            is_prime[num] = false;
            min_prime[num] = p;
            if(i % p == 0) {
                phi[num] = phi[i] * p;
                mu[num] = 0;
                cnt_min_p[num] = cnt_min_p[i] + 1;
                d[num] = d[i] / (cnt_min_p[i] + 1) * (cnt_min_p[i] + 1);
                break;
            } else {
                phi[num] = phi[i] * (p - 1);
                mu[num] = -mu[i];
                cnt_min_p[num] = 1;
                d[num] = d[i] * 2;
            }
        }
    }
}
\end{lstlisting}

\subsection{Comb(结合ModInt)}
\begin{lstlisting}
struct Comb {
    int n;
    std::vector<Z> _fac;
    std::vector<Z> _invfac;
    std::vector<Z> _inv;
    
    Comb() : n{0}, _fac{1}, _invfac{1}, _inv{0} {}
    Comb(int n) : Comb() {
        init(n);
    }
    
    void init(int m) {
        if (m <= n) return;
        _fac.resize(m + 1);
        _invfac.resize(m + 1);
        _inv.resize(m + 1);
        
        for (int i = n + 1; i <= m; i++) {
            _fac[i] = _fac[i - 1] * i;
        }
        _invfac[m] = _fac[m].inv();
        for (int i = m; i > n; i--) {
            _invfac[i - 1] = _invfac[i] * i;
            _inv[i] = _invfac[i] * _fac[i - 1];
        }
        n = m;
    }
    
    Z fac(int m) {
        if (m > n) init(2 * m);
        return _fac[m];
    }
    Z invfac(int m) {
        if (m > n) init(2 * m);
        return _invfac[m];
    }
    Z inv(int m) {
        if (m > n) init(2 * m);
        return _inv[m];
    }
    Z C(int n, int m) {
        if (n < m || m < 0) return 0;
        return fac(n) * invfac(m) * invfac(n - m);
    }
    Z A(int n, int m) {
        if(n < m || m < 0) return 0;
        return fac(n) * invfac(n - m);
    }
} comb;
\end{lstlisting}


\subsection{BigInt}
\begin{lstlisting}
constexpr int BASE = 100000000;
constexpr int MAX_LENGTH = 1000;
constexpr int NUM_DIGIT = 8;
i64 mul_mod (i64 x, i64 y, i64 n){
	i64 T = std::floor(std::sqrt(n) + 0.5);
	i64 t = T * T - n;
	i64 a = x / T; i64 b = x % T;
	i64 c = y / T; i64 d = y % T;
	i64 e = a * c / T; i64 f = a * c % T;
	i64 v = ((a * d + b * c) % n + e * t) % n;
	i64 g = v / T; i64 h = v % T;
	i64 ans = (((f + g) * t % n + b * d) % n + h * T) % n;
	while (ans < 0) {
		ans += n;
	}
	return ans;
}
struct BigInt {

	BigInt(const char* str = "0") {
		(*this) = str;
	}

	BigInt operator=(const char* str) {
		int j = std::strlen(str) - 1;
		len = j / NUM_DIGIT + 1;
		for(int i = 0; i <= len; i++) {
			s[i] = 0;
		}
		for(int i = 0; i <= j; i++) {
			int k = (j - i) / NUM_DIGIT + 1;
			s[k] = s[k] * 10 + (str[i] - '0');
		}
		return *this;
	}

	void print() {
		printf("%d", s[len]);
		for(int i = len - 1; i >= 1; i--) {
			printf("%0*d", NUM_DIGIT, s[i]);
		}
	}

	int len;
	int s[MAX_LENGTH];
};

// > return > 0
// < return < 0
// = return = 0
int compare(const BigInt &lhs, const BigInt &rhs) {
	if(lhs.len > rhs.len) {
		return 1;
	}
	if(lhs.len < rhs.len) {
		return -1;
	}
	int cur = lhs.len;
	while((cur > 1) && (lhs.s[cur] == rhs.s[cur])) {
		cur--;
	}
	return lhs.s[cur] - rhs.s[cur];
}

bool operator<(const BigInt &lhs, const BigInt &rhs) {
	return compare(lhs, rhs) < 0;
}

bool operator<=(const BigInt &lhs, const BigInt &rhs) {
	return compare(lhs, rhs) <= 0;
}

bool operator>(const BigInt &lhs, const BigInt &rhs) {
	return compare(lhs, rhs) > 0;
}

bool operator>=(const BigInt &lhs, const BigInt &rhs) {
	return compare(lhs, rhs) >= 0;
}

bool operator==(const BigInt &lhs, const BigInt &rhs) {
	return compare(lhs, rhs) == 0;
}

bool operator!=(const BigInt &lhs, const BigInt &rhs) {
	return compare(lhs, rhs) != 0;
}


BigInt operator+(const BigInt &lhs, const BigInt &rhs) {
	BigInt ret;
	int i;
	for(i = 1; i <= lhs.len || i <= rhs.len || ret.s[i]; i++) {
		if(i <= lhs.len) {
			ret.s[i] += lhs.s[i];
		}
		if(i <= rhs.len) {
			ret.s[i] += rhs.s[i];
		}
		ret.s[i + 1] = ret.s[i] / BASE;
		ret.s[i] %= BASE;
	}
	ret.len = i - 1;
	if(ret.len == 0) {
		ret.len = 1;
	}
	return ret;
}

// lhs > rhs
BigInt operator-(const BigInt &lhs, const BigInt &rhs) {
	BigInt ret;
	for(int i = 1, j = 0; i <= lhs.len; i++) {
		ret.s[i] = lhs.s[i] - j;
		if(i <= rhs.len) {
			ret.s[i] -= rhs.s[i];
		}
		if(ret.s[i] < 0) {
			j = 1;
			ret.s[i] += BASE;
		} else {
			j = 0;
		}
	}
	ret.len = lhs.len;
	while(ret.len > 1 && !ret.s[ret.len]) {
		ret.len--;
	}
	return ret;
}

BigInt operator*(const BigInt &lhs, const BigInt &rhs) {
	BigInt ret;
	i64 g = 0;
	ret.len = lhs.len + rhs.len;
	ret.s[0] = 0;
	for(int i = 1; i <= ret.len; i++) {
		ret.s[i] = 0;
	}
	for(int k = 1; k <= ret.len; k++) {
		i64 tmp = g;
		int j = k + 1 - rhs.len;
		if(j < 1) {
			j = 1;
		}
		for(; j <= k && j <= lhs.len; j++) {
			tmp += (i64)lhs.s[j] * (i64)rhs.s[k + 1 - j];
		}
		g = tmp / BASE;
		ret.s[k] = tmp % BASE;
	}
	while(ret.len > 1 && !ret.s[ret.len]) {
		ret.len--;
	}
	return ret;
}
BigInt operator/(const BigInt &lhs, int num) {
	i64 g = 0;
	BigInt ret;
	ret.len = lhs.len;
	for(int i = lhs.len; i > 0; i--) {
		i64 tmp = g * BASE + lhs.s[i];
		ret.s[i] = tmp / num;
		g = tmp % num;
	}
	while(ret.len > 1 && ! ret.s[ret.len]) {
		ret.len--;
	}
	return ret;
}
BigInt operator/(const BigInt &lhs, const BigInt &rhs) {
	BigInt l = "0", r = lhs;
	while(l < r) {
		BigInt m = l + (r - l + "1") / 2;
		if(m * rhs <= lhs) {
			l = m;
		} else {
			r = m - "1";
		}
	}
	return l;
}

i64 BigMod(const BigInt &a, i64 m) {
	i64 d = 0;
	for(int i = a.len; i > 0; i--) {
		d = mul_mod(d, BASE, m);
		d = (d + a.s[i]) % m;
	}
	return d;
}
BigInt sqrt(const BigInt &a) {
	BigInt x, y = a;
	do {
		x = y;
		y = (x + a / x) / 2;
	}while(y < x);

	return x;
}
BigInt gcd(BigInt a, BigInt b) {
	BigInt c = "1";
	while(true) {
		if(a == b) {
			return a * c;
		} else if(a.s[1] % 2 == 0 && b.s[1] % 2 == 0) {
			a = a / 2;
			b = b / 2;
			c = c * "2";
		} else if(a.s[1] % 2 == 0) {
			a = a / 2;
		} else if(b.s[1] % 2 == 0) {
			b = b / 2;
		} else if(b < a) {
			a = a - b;
		} else {
			b = b - a;
		}
	}
}
\end{lstlisting}

\subsection{Miller-RabinAndPollard-Rho}
\begin{lstlisting}
i64 mul(i64 a, i64 b, i64 m) {
    return static_cast<__int128>(a) * b % m;
}
i64 power(i64 a, i64 b, i64 m) {
    i64 res = 1 % m;
    for (; b; b >>= 1, a = mul(a, a, m))
        if (b & 1)
            res = mul(res, a, m);
    return res;
}
bool isprime(i64 n) {
    if (n < 2)
        return false;
    static constexpr int A[] = {2, 3, 5, 7, 11, 13, 17, 19, 23};
    int s = __builtin_ctzll(n - 1);
    i64 d = (n - 1) >> s;
    for (auto a : A) {
        if (a == n)
            return true;
        i64 x = power(a, d, n);
        if (x == 1 || x == n - 1)
            continue;
        bool ok = false;
        for (int i = 0; i < s - 1; ++i) {
            x = mul(x, x, n);
            if (x == n - 1) {
                ok = true;
                break;
            }
        }
        if (!ok)
            return false;
    }
    return true;
}
std::vector<i64> factorize(i64 n) {
    std::vector<i64> p;
    std::function<void(i64)> f = [&](i64 n) {
        if (n <= 10000) {
            for (int i = 2; i * i <= n; ++i)
                for (; n % i == 0; n /= i)
                    p.push_back(i);
            if (n > 1)
                p.push_back(n);
            return;
        }
        if (isprime(n)) {
            p.push_back(n);
            return;
        }
        auto g = [&](i64 x) {
            return (mul(x, x, n) + 1) % n;
        };
        i64 x0 = 2;
        while (true) {
            i64 x = x0;
            i64 y = x0;
            i64 d = 1;
            i64 power = 1, lam = 0;
            i64 v = 1;
            while (d == 1) {
                y = g(y);
                ++lam;
                v = mul(v, std::abs(x - y), n);
                if (lam % 127 == 0) {
                    d = std::gcd(v, n);
                    v = 1;
                }
                if (power == lam) {
                    x = y;
                    power *= 2;
                    lam = 0;
                    d = std::gcd(v, n);
                    v = 1;
                }
            }
            if (d != n) {
                f(d);
                f(n / d);
                return;
            }
            ++x0;
        }
    };
    f(n);
    std::sort(p.begin(), p.end());
    return p;
}
\end{lstlisting}
