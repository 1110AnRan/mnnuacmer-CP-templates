\section{数据结构专题模板}

\subsection{普通并查集}
\begin{lstlisting}
struct DSU {
	std::vector<int> f, siz;

	DSU() {}
	DSU(int n){
		init(n);
	}
	
	void init(int n){
		f.resize(n);
		std::iota(f.begin(), f.end(), 0);
		siz.assign(n, 1);
	}

	int find(int x){
		while(x != f[x]){
			x = f[x] = f[f[x]];
		}
		return x;
	}

	bool same(int x,int y){
		return find(x) == find(y);
	}

	bool merge(int x,int y){
		x = find(x);
		y = find(y);
		if(x == y){
			return false;
		}
		siz[x] += siz[y];
		f[y] = x;
		return true;
	}

	int size(int x){
		return siz[find(x)];
	}

    int operator[](const int x) {
		return find(x);
	}
};
\end{lstlisting}


\subsection{带权并查集}
\begin{lstlisting}
struct WDSU {
    std::vector<int> dist, f;

    WDSU() {}

    WDSU(int n) {
        init(n);
    }

    void init(int n) {
        f.resize(n);
        std::iota(f.begin(), f.end(), 0);
        dist.assign(n, 0);
    }

    int find(int x) {
        while(x != f[x]) {
            int tmp = f[x];
            f[x] = find(tmp);
            dist[x] += dist[tmp];
        }
        return f[x];
    }

    bool merge(int l, int r, int v) {
        int lf = find(l), rf = find(r);
        if(lf != rf) {
            f[lf] = rf;
            dist[lf] = v + dist[r] - dist[l];
        }
    }

    int query(int l, int r) {
        if(find(l) != find(r)) {
            return -1;
        }
        return dist[l] - dist[r];
    }
    int operator[](const int x) {
        return find(x);
    }
};
\end{lstlisting}



\subsection{可撤销并查集}
\begin{lstlisting}
struct UndoDSU {
	std::vector<int> f, siz, rank;
    std::stack<std::pair<int, int>> stk;

	DSU() {}
	DSU(int n){
		init(n);
	}
	
	void init(int n){
		f.resize(n);
		std::iota(f.begin(), f.end(), 0);
		siz.assign(n, 1);
        rank.assign(n, 0);
	}

	int find(int x){
		while(x != f[x]){
			x = f[x];
		}
		return x;
	}

	bool same(int x,int y){
		return find(x) == find(y);
	}

	bool merge(int x,int y){
		x = find(x);
		y = find(y);
		if(x == y){
            stk.push({-1, -1});
			return false;
		}
        if(rank[x] > rank[y]) {
            std::swap(x, y);
        }

        f[x] = y;
        siz[y] += siz[x];
        stk.push({x, 0});
        if(rank[x] == rank[y]) {
            rank[y]++;
            stk.top().second = 1;
        }

	}

    void rollback() {
        auto v = stk.top();
        stk.pop();
        if(v.first == -1) {
            return;
        }
        int x = v.first, y = f[x];
        rank[y] -= v.second;
        siz[y] -= siz[x];
        f[x] = x;
    }

	int size(int x){
		return siz[find(x)];
	}
    int operator[](const int x) {
        return find(x);
    }
};
\end{lstlisting}



\subsection{树状数组}
\begin{lstlisting}
template <typename T>
struct Fenwick {
    int n;
    std::vector<T> a;
    
    Fenwick(int n_ = 0) {
        init(n_);
    }
    
    void init(int n_) {
        n = n_;
        a.assign(n, T{});
    }
    
    void add(int x, const T &v) {
        for (int i = x + 1; i <= n; i += i & -i) {
            a[i - 1] = a[i - 1] + v;
        }
    }
    
    T sum(int x) {
        T ans{};
        for (int i = x; i > 0; i -= i & -i) {
            ans = ans + a[i - 1];
        }
        return ans;
    }
    
    T rangeSum(int l, int r) {
        return sum(r) - sum(l);
    }
    
    // first p, query(0, p) >= k
    int select(const T &k) {
        int x = 0;
        T cur{};
        for (int i = 1 << std::__lg(n); i; i /= 2) {
            if (x + i <= n && cur + a[x + i - 1] <= k) {
                x += i;
                cur = cur + a[x - 1];
            }
        }
        return x;
    }
};
\end{lstlisting}



\subsection{二维树状数组}
\begin{lstlisting}
struct BIT_2D {
    using T = long long;
    int n, m;
    vector<vector<T>> sum[4];
    BIT_2D(int _n, int _m):n(_n), m(_m) {
        for (int i=0; i<4; ++i) sum[i].assign(n+1, vector<T>(m+1, 0));
    }
    void add(int x, int y, T val) {
        for (int i=x; i<=n; i+=i&-i) {
            for (int j=y; j<=m; j+=j&-j) {
                sum[0][i][j]+=val;
                sum[1][i][j]+=val*x;
                sum[2][i][j]+=val*y;
                sum[3][i][j]+=val*x*y;
            }
        }
    }
    void range_add(int x1, int y1, int x2, int y2, T x) {
        add(x1, y1, x);
        add(x1, y2+1, -x);
        add(x2+1, y1, -x);
        add(x2+1, y2+1, x);
    }
    T ask(int x, int y) {
        T res[4]= {};
        for (int i=x; i>0; i-=i&-i)
            for (int j=y; j>0; j-=j&-j)
                for (int k=0; k<4; ++k)
                    res[k]+=sum[k][i][j];
        return (x+1)*(y+1)*res[0]-(y+1)*res[1]-(x+1)*res[2]+res[3];
    }
    T range_ask(int x1, int y1, int x2, int y2) {
        return ask(x2, y2)-ask(x1-1, y2)-ask(x2, y1-1)+ask(x1-1, y1-1);
    }
};
\end{lstlisting}



\subsection{线段树}
\subsubsection{无lazy}
\begin{lstlisting}
template<class Info>
struct SegmentTree {
    int n;
    std::vector<Info> info;
    SegmentTree() : n(0) {}
    SegmentTree(int n_, Info v_ = Info()) {
        init(n_, v_);
    }
    template<class T>
    SegmentTree(std::vector<T> init_) {
        init(init_);
    }
    void init(int n_, Info v_ = Info()) {
        init(std::vector(n_, v_));
    }
    template<class T>
    void init(std::vector<T> init_) {
        n = init_.size();
        info.assign(4 << std::__lg(n), Info());
        std::function<void(int, int, int)> build = [&](int p, int l, int r) {
            if (r - l == 1) {
                info[p] = init_[l];
                return;
            }
            int m = (l + r) / 2;
            build(2 * p, l, m);
            build(2 * p + 1, m, r);
            pull(p);
        };
        build(1, 0, n);
    }
    void pull(int p) {
        info[p] = info[2 * p] + info[2 * p + 1];
    }
    void modify(int p, int l, int r, int x, const Info &v) {
        if (r - l == 1) {
            info[p] = v;
            return;
        }
        int m = (l + r) / 2;
        if (x < m) {
            modify(2 * p, l, m, x, v);
        } else {
            modify(2 * p + 1, m, r, x, v);
        }
        pull(p);
    }
    void modify(int p, const Info &v) {
        modify(1, 0, n, p, v);
    }
    Info rangeQuery(int p, int l, int r, int x, int y) {
        if (l >= y || r <= x) {
            return Info();
        }
        if (l >= x && r <= y) {
            return info[p];
        }
        int m = (l + r) / 2;
        return rangeQuery(2 * p, l, m, x, y) \
            + \
            rangeQuery(2 * p + 1, m, r, x, y);
    }
    Info rangeQuery(int l, int r) {
        return rangeQuery(1, 0, n, l, r);
    }
    template<class F>
    int findFirst(int p, int l, int r, int x, int y, F &&pred) {
        if (l >= y || r <= x) {
            return -1;
        }
        if (l >= x && r <= y && !pred(info[p])) {
            return -1;
        }
        if (r - l == 1) {
            return l;
        }
        int m = (l + r) / 2;
        int res = findFirst(2 * p, l, m, x, y, pred);
        if (res == -1) {
            res = findFirst(2 * p + 1, m, r, x, y, pred);
        }
        return res;
    }
    template<class F>
    int findFirst(int l, int r, F &&pred) {
        return findFirst(1, 0, n, l, r, pred);
    }
    template<class F>
    int findLast(int p, int l, int r, int x, int y, F &&pred) {
        if (l >= y || r <= x) {
            return -1;
        }
        if (l >= x && r <= y && !pred(info[p])) {
            return -1;
        }
        if (r - l == 1) {
            return l;
        }
        int m = (l + r) / 2;
        int res = findLast(2 * p + 1, m, r, x, y, pred);
        if (res == -1) {
            res = findLast(2 * p, l, m, x, y, pred);
        }
        return res;
    }
    template<class F>
    int findLast(int l, int r, F &&pred) {
        return findLast(1, 0, n, l, r, pred);
    }
};
\end{lstlisting}



\subsubsection{有lazy}
\begin{lstlisting}
template<class Info, class Tag>
struct LazySegmentTree {
    int n;
    std::vector<Info> info;
    std::vector<Tag> tag;
    LazySegmentTree() : n(0) {}
    LazySegmentTree(int n_, Info v_ = Info()) {
        init(n_, v_);
    }
    template<class T>
    LazySegmentTree(std::vector<T> init_) {
        init(init_);
    }
    void init(int n_, Info v_ = Info()) {
        init(std::vector(n_, v_));
    }
    template<class T>
    void init(std::vector<T> init_) {
        n = init_.size();
        info.assign(4 << std::__lg(n), Info());
        tag.assign(4 << std::__lg(n), Tag());
        std::function<void(int, int, int)> build = [&](int p, int l, int r) {
            if (r - l == 1) {
                info[p] = init_[l];
                return;
            }
            int m = (l + r) / 2;
            build(2 * p, l, m);
            build(2 * p + 1, m, r);
            pull(p);
        };
        build(1, 0, n);
    }
    void pull(int p) {
        info[p] = info[2 * p] + info[2 * p + 1];
    }
    void apply(int p, const Tag &v) {
        info[p].apply(v);
        tag[p].apply(v);
    }
    void push(int p) {
        apply(2 * p, tag[p]);
        apply(2 * p + 1, tag[p]);
        tag[p] = Tag();
    }
    void modify(int p, int l, int r, int x, const Info &v) {
        if (r - l == 1) {
            info[p] = v;
            return;
        }
        int m = (l + r) / 2;
        push(p);
        if (x < m) {
            modify(2 * p, l, m, x, v);
        } else {
            modify(2 * p + 1, m, r, x, v);
        }
        pull(p);
    }
    void modify(int p, const Info &v) {
        modify(1, 0, n, p, v);
    }
    Info rangeQuery(int p, int l, int r, int x, int y) {
        if (l >= y || r <= x) {
            return Info();
        }
        if (l >= x && r <= y) {
            return info[p];
        }
        int m = (l + r) / 2;
        push(p);
        return rangeQuery(2 * p, l, m, x, y) \ 
                + \
                rangeQuery(2 * p + 1, m, r, x, y);
    }
    Info rangeQuery(int l, int r) {
        return rangeQuery(1, 0, n, l, r);
    }
    void rangeApply(int p, int l, int r, int x, int y, const Tag &v) {
        if (l >= y || r <= x) {
            return;
        }
        if (l >= x && r <= y) {
            apply(p, v);
            return;
        }
        int m = (l + r) / 2;
        push(p);
        rangeApply(2 * p, l, m, x, y, v);
        rangeApply(2 * p + 1, m, r, x, y, v);
        pull(p);
    }
    void rangeApply(int l, int r, const Tag &v) {
        return rangeApply(1, 0, n, l, r, v);
    }
    template<class F>
    int findFirst(int p, int l, int r, int x, int y, F pred) {
        if (l >= y || r <= x || !pred(info[p])) {
            return -1;
        }
        if (r - l == 1) {
            return l;
        }
        int m = (l + r) / 2;
        push(p);
        int res = findFirst(2 * p, l, m, x, y, pred);
        if (res == -1) {
            res = findFirst(2 * p + 1, m, r, x, y, pred);
        }
        return res;
    }
    template<class F>
    int findFirst(int l, int r, F pred) {
        return findFirst(1, 0, n, l, r, pred);
    }
    template<class F>
    int findLast(int p, int l, int r, int x, int y, F pred) {
        if (l >= y || r <= x || !pred(info[p])) {
            return -1;
        }
        if (r - l == 1) {
            return l;
        }
        int m = (l + r) / 2;
        push(p);
        int res = findLast(2 * p + 1, m, r, x, y, pred);
        if (res == -1) {
            res = findLast(2 * p, l, m, x, y, pred);
        }
        return res;
    }
    template<class F>
    int findLast(int l, int r, F pred) {
        return findLast(1, 0, n, l, r, pred);
    }
};
\end{lstlisting}



\subsection{ST表}
\begin{lstlisting}
// vector<int> a(n + 1);
template<typename T>
class SparseTable {
public:
    SparseTable() = default;
    
    explicit SparseTable(const std::vector<T>& data) 
    {
        Initialize(data);
    }

    void Initialize(const std::vector<T>& data) {
        this->n = data.size() - 1;

        log_table.resize(n + 1);
        log_table[0] = -1;
        for(int i = 1; i <= n; i++) {
            log_table[i] = log_table[i >> 1] + 1;
        }

        st_table.resize(n + 1, std::vector<int>(21));
        
        for(int i = 1; i <= n; i++) {
            st_table[i][0] = data[i];
        }

        for(int p = 1; p <= log_table[n]; p++) {
            for(int i = 1; i + (1 << p) - 1 <= n; i++) {
                st_table[i][p] = op(st_table[i][p - 1], 
                    st_table[i + (1 << (p - 1))][p - 1]);
            }
        }

    }

    T Query(size_t left, size_t right) {
        const int k = log_table[right - left + 1];
        return op(st_table[left][k], st_table[right - (1 << k) + 1][k]);
    }

private:
    int n;
    std::vector<int> log_table;
    std::vector<std::vector<T>> st_table;

    T op(const T& lv, const T& rv) const {
        return std::max(lv, rv);
    }
};
\end{lstlisting}

