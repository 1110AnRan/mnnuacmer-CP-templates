\section{图论专题模板}
\subsection{链式前向星建图}
\begin{lstlisting}
const int N = 2e6 + 10;
struct Enode {
    int next, to, w;
}edges[N];
// 如果是双向图,那么edges中的N为 N << 1;
int head[N], cnt = 0;

void init() {
    std::fill(head, head + N, -1);
    cnt = 0;
}

void addEdge(int u, int v, int w = 1) {
    edges[cnt].to = v;
    edges[cnt].w = w;
    edges[cnt].next = head[u];
    head[u] = cnt++;
}
\end{lstlisting}



\subsection{Dijkstra}
\begin{lstlisting}
struct Dijkstra {   // index-base-0
    using i64 = long long;
    int n;
    std::vector<std::vector<std::pair<int, i64>>> adj;
    std::vector<i64> dis;
    std::vector<bool> vis;
    Dijkstra(int n) {
        init(n);
    }

    void init(int n) {
        this->n = n;
        adj.assign(n, {});
        vis.assign(n, false);
        dis.assign(n, 1e18);
    }
 
    void add_edge(int u, int v, i64 d) {
        adj[u].push_back({v, d});
        adj[v].push_back({u, d});
    }
 

    struct node {
        i64 dis;
        int pos;
        bool operator <(const node &x) const {
            return x.dis < dis;
        }
    };
 
    std::priority_queue<node> q;
    void work(int r) {
        dis[r] = 0;
        q.push((node){0, r});
 
        while(!q.empty()) {
            node tmp = q.top();
            q.pop();
            int u = tmp.pos;
            i64 d = tmp.dis;
 
            if (vis[u]) {
                continue;
            }
 
            vis[u] = true;
            for(auto [v, w] : adj[u]) {
                if(dis[v] > dis[u] + w) {
                    dis[v] = dis[u] + w;
                    q.push((node){dis[v], v});
                }
            }
        }
    }
};
\end{lstlisting}

\subsection{Spfa}
\begin{lstlisting}
template<class T = long long>
struct Spfa {   // index-base-0
	static constexpr T inf = std::numeric_limits<T>::max() / 2;

	int n;
	std::vector<std::vector<std::pair<int, T>>> g;
	std::vector<T> dis;

	Spfa() {};

	Spfa(int n) : n(n), g(n) {}

	void add_Edge(int u, int v, T w) {
        g[u].emplace_back({v, w});
	}

	bool work(int root = 0) {
		std::vector<bool> vis(n, false);
		std::vector<int> cnt(n, 0);
        dis.assign(n, inf);
		dis[root] = 0;
        // 最长路
        // dis.assign(n, -inf);
		std::queue<int> q;
		q.push(root);
		vis[root] = true;
		while(q.size()) {
			auto u = q.front();
			q.pop();
			vis[u] = false;
			for(auto [v, w] : g[u]) {
                if(dis[v] > dis[u] + w) { // 最长路 -> v < u + w
                    dis[v] = dis[u] + w;
                    if((cnt[v] = cnt[u] + 1) >= n) {
                        return false;
                    }
                    if(!vis[v]) {
                        q.push(v);
                        vis[v] = true;
                    }
                }
			}
		}
		return true;
	}
};
\end{lstlisting}



\subsection{Floyd}
\begin{lstlisting}
const int inf = 1e9;
void floyd(std::vector<std::vector<int>>& dis) {
    int n = dis.size();
    for(int k=0; k<n; ++k)
        for(int i=0; i<n; ++i)
            for(int j=0; j<n; ++j) {
                if(i!=j && i!=k && j!=k 
                && dis[i][k] < inf && dis[k][j] < inf// 边权有负数必加
                && dis[i][j] > dis[i][k] + dis[k][j])
                    dis[i][j] = dis[i][k]+dis[k][j];
            }
}
// 同时记录方案数
void floyd(std::vector<std::vector<int>>& dis, std::vector<std::vector<Z>>& f) {
    int n = dis.size();
    for(int k=0; k<n; ++k)
        for(int i=0; i<n; ++i)
            for(int j=0; j<n; ++j)
                if(i!=j && i!=k && j!=k) {
                    if(dis[i][j] > dis[i][k] + dis[k][j]) {
                        dis[i][j] = dis[i][k] + dis[k][j];
                        f[i][i] = f[i][k] * f[k][j];
                    }else if(dis[i][j] == dis[i][k] + dis[k][j]) {
                        f[i][j] += f[i][k] * f[k][j];
                    }
                }
}
\end{lstlisting}


\subsection{Kruskal}
\begin{lstlisting}
template<class T>
struct KruskalMst {		// index-base-0
    int n;
    std::vector<int> f;
    std::vector<std::tuple<T, int, int>> e;
    KruskalMst(int n) {
		this->n = n;
		f.resize(n);
		std::iota(f.begin(), f.end(), 0);
	}
    void addEdge(int u, int v, T w) {
        e.emplace_back(w, u, v);
    }
    int find(int u) {
        return f[u] == u ? u : f[u] = find(f[u]);
    }
    T work() {
        std::sort(e.begin(), e.end());
        T ans = 0;
        int cnt = 0;
		for(auto [w, u, v] : e) {
			u = find(u);
			v = find(v);
			if(u == v) {
				continue;
			}
			f[u] = v;
			ans += w;
			if(++cnt == n - 1) {
				break;
			}
		}
        // assert(cnt==n-1);
        return ans;
    }
};
\end{lstlisting}


\subsection{二分图最大匹配}
\subsubsection{匈牙利算法}
\begin{lstlisting}
struct BipartiteGraph {		// index-base-0
    int n, m;
    std::vector<std::vector<int>> g;
    std::vector<int> vis, link;
    BipartiteGraph(int n,int m)
	:n(n),m(m), g(n), vis(m), link(m) {}

    void addEdge(int u, int v) {// left->right
        g[u].push_back(v);
    }
    bool find(int u) {// 左边u -> 右边v
        for(int v: g[u]) {
            if(vis[v]) {
				continue;
			}
            vis[v] = 1;
            if(link[v] == -1 || find(link[v])) {
                link[v] = u;
                return true;
            }
        }
        return false;
    }
    int maxMatching() {
        int cnt = 0;
        std::fill_n(link.begin(), m, -1);
        for(int i = 0; i < n; ++i) {
            if(!find(i)) {
				continue;
			}
            std::fill_n(vis.begin(), m, 0);
            ++cnt;
        }
        return cnt;
    }
};
\end{lstlisting}


\subsubsection{HK算法}
\begin{lstlisting}
struct HopcroftKarp {		// index-base-0
    static constexpr int INF = 0x3f3f3f3f;
    int n, m, dis;
    std::vector<std::vector<int>> e;
    std::vector<int> matchX, matchY, dx, dy;
    std::vector<bool> used;
    HopcroftKarp(int n, int m)
    :n(n), m(m), e(n), matchX(n), matchY(m), dx(n), dy(m), used(m) {}
    void addEdge(int u, int v) {
        e[u].push_back(v);
    }
    bool searchP() {
        std::fill(dx.begin(), dx.end(), -1);
        std::fill(dy.begin(), dy.end(), -1);
        dis = INF;
        std::queue<int> q;
        for(int i = 0; i < n; ++i) {
            if(matchX[i] == -1) {
                q.push(i), dx[i] = 0;
            }
        }
        while(!q.empty()) {
            int u = q.front(); 
            q.pop();
            if(dx[u] > dis) {
				break;
			}
            for(int v : e[u])
                if(dy[v] == -1) {
                    dy[v] = dx[u] + 1;
                    if(matchY[v] == -1) {
						dis = dy[v];
					}
                    else {
						dx[matchY[v]] = dy[v] + 1;
						q.push(matchY[v]);
					}
                }
        }
        return dis != INF;
    }
    bool dfs(int u) {
        for(int v : e[u]) {
            if(used[v] || dy[v] != dx[u] + 1) {
				continue;
			}
            used[v] = true;
            if(matchY[v] != -1 && dy[v] == dis) {
				continue;
			}
            if(matchY[v] == -1 || dfs(matchY[v])) {
                matchY[v] = u;
                matchX[u] = v;
                return true;
            }
        }
        return false;
    }
    int maxMatching() {
        int res = 0;
        std::fill(matchX.begin(), matchX.end(), -1);
        std::fill(matchY.begin(), matchY.end(), -1);
        while(searchP()) {
            std::fill(used.begin(), used.end(), false);
            for(int i = 0; i < n; ++i) {
				if(matchX[i] == -1 && dfs(i)) {
					++res;
				}
			}
        }
        return res;
    }
};
\end{lstlisting}


\subsection{二分图判定}
\begin{lstlisting}
struct JudgeBG {	// index-base-0
    int n;
    std::vector<int> bel;	// bel:1 or 2
    std::vector<std::vector<int>> g;
    JudgeBG(int n): n(n),g(n),bel(n) {}
    void addEdge(int u, int v) {
        g[u].push_back(v);
        g[v].push_back(u);
    }
    bool dfs(int u, int color = 1) {
        if(bel[u]) {
			return bel[u]==color;
		}
        bel[u] = color;
        for(int v : g[u])
            if(!dfs(v, 3-color)) {
				return false;
			} 
        return true;
    }
    bool paint() {// 二分图染色
        for(int i = 0; i < n; ++i) {
            if(bel[i]) {
				continue;
			}
            if(!dfs(i)) {
				return false;
			}
        }
        return true;
    }
    // std::vector<int> to;//[0,ln), [0,rn)
    // //  use it to divide the graph if paint()=true
    // BipartiteGraph reLabel() {// left->right
    //     int ln = 0,rn = 0;
    //     to.resize(n);
    //     for(int i = 0; i < n; ++i) {
    //         if(bel[i] == 1) {
	// 			to[i] = ln++;
	// 		}
    //         else {
	// 			to[i] = rn++;
	// 		}
    //     }
    //     BipartiteGraph bg(ln, rn);
    //     for(int i = 0; i < n; ++i) {
    //         for(int j : g[i]) {
    //             if(bel[i] == 1) {
	// 				bg.addEdge(to[i], to[j]); 
	// 			}
	// 		}
	// 	}
    //     return bg;
    // }
};
\end{lstlisting}


\subsection{Kruskal重构树}
\begin{lstlisting}
namespace krt {     // index-1
    constexpr int MAXN = 2e5 + 10, MAXH = 21;
    constexpr bool ASC = true;  // 边权从小到大排序
    int father[MAXN], head[MAXN], next[MAXN], to[MAXN];
    int nodekey[MAXN], dep[MAXN], stjump[MAXN][MAXH];
    int /*dfn[MAXN], */seg[MAXN], in[MAXN], out[MAXN];
    int cntu, cntg, cntd;
    int find(int x) {
        return x == father[x] ? x : father[x] = find(father[x]);
    }

    void addEdge(int u, int v) {
        next[++cntg] = head[u];
		to[cntg] = v;
		head[u] = cntg;
    }

    int lca(int x, int y) {
        if(dep[x] < dep[y]) {
            std::swap(x, y);
        }
        for(int p = MAXH - 1; p >= 0; p--) {
            if(dep[stjump[x][p]] >= dep[y]) {
                x = stjump[x][p];
            }
        }
        if(x == y) {
            return x;
        }
        for(int p = MAXH - 1; p >= 0; p--) {
            if(stjump[x][p] != stjump[y][p]) {
                x = stjump[x][p];
                y = stjump[y][p];
            }
        }
        return stjump[x][0];
    }

    void build(int n, std::vector<std::tuple<int, int, int>> e) {
        for(int i = 1; i <= n; i++) {
            father[i] = i;
        }
        if constexpr (ASC) {
            std::sort(e.begin(), e.end());
        } else {
            std::sort(e.begin(), e.end(), std::greater());
        }
        cntu = n;
        for(auto& [w, u, v] : e) {
            int fx = find(u);
            int fy = find(v);
            if(fx != fy) {
                father[fx] = father[fy] = ++cntu;
                father[cntu] = cntu;
                nodekey[cntu] = w;
                addEdge(cntu, fx);
                addEdge(cntu, fy);
            }
        }
        std::function<void(int, int)> dfs1 = [&](int u, int fa) -> void {
            dep[u] = dep[fa] + 1;
            in[u] = ++cntd;
            // dfn[u] = ++cntd;
            seg[cntd] = u;
            stjump[u][0] = fa;
            for(int i = 1; i < MAXH; i++) {
                stjump[u][i] = stjump[stjump[u][i - 1]][i - 1];
            }
            for(int e = head[u]; e > 0; e = next[e]) {
                dfs1(to[e], u);
            }
            out[u] = cntd;
        };
        for(int i = 1; i <= cntu; i++) {
            if(i == father[i]) {
                dfs1(i, 0);
            }
        }
    }
    // 在lim的限制下最高能跳到哪一个节点
    int jumpUp(int u, int lim) {
        for(int i = MAXH - 1; i >= 0; --i) {
            if constexpr (ASC) {
                if(stjump[i][u] and nodekey[stjump[i][u]]<=lim) { 
                    u = stjump[i][u];
                }
            } else {
                if(stjump[i][u] and nodekey[stjump[i][u]]>=lim) {
                    u = stjump[i][u];
                }
            }
        }
        return u;
    }
}
\end{lstlisting}

\subsection{SCC点-双连通分量}
\begin{lstlisting}
// index-base-0
struct SCC {
    int n;
    std::vector<std::vector<int>> adj;
    std::vector<int> stk;
    std::vector<int> dfn, low, bel;
    int cur, cnt;
    
    SCC() {}
    SCC(int n) {
        init(n);
    }
    
    void init(int n) {
        this->n = n;
        adj.assign(n, {});
        dfn.assign(n, -1);
        low.resize(n);
        bel.assign(n, -1);
        stk.clear();
        cur = cnt = 0;
    }
    
    void addEdge(int u, int v) {
        adj[u].push_back(v);
    }
    
    void dfs(int x) {
        dfn[x] = low[x] = cur++;
        stk.push_back(x);
        
        for (auto y : adj[x]) {
            if (dfn[y] == -1) {
                dfs(y);
                low[x] = std::min(low[x], low[y]);
            } else if (bel[y] == -1) {
                low[x] = std::min(low[x], dfn[y]);
            }
        }
        
        if (dfn[x] == low[x]) {
            int y;
            do {
                y = stk.back();
                bel[y] = cnt;
                stk.pop_back();
            } while (y != x);
            cnt++;
        }
    }
    
    std::vector<int> work() {
        for (int i = 0; i < n; i++) {
            if (dfn[i] == -1) {
                dfs(i);
            }
        }
        return bel;
    }
};
\end{lstlisting}


\subsection{EBCC边-双连通分量}
\begin{lstlisting}
std::set<std::pair<int, int>> E;
 
struct EBCC {
    int n;
    std::vector<std::vector<int>> adj;
    std::vector<int> stk;
    std::vector<int> dfn, low, bel;
    int cur, cnt;
    
    EBCC() {}
    EBCC(int n) {
        init(n);
    }
    
    void init(int n) {
        this->n = n;
        adj.assign(n, {});
        dfn.assign(n, -1);
        low.resize(n);
        bel.assign(n, -1);
        stk.clear();
        cur = cnt = 0;
    }
    
    void addEdge(int u, int v) {
        adj[u].push_back(v);
        adj[v].push_back(u);
    }
    
    void dfs(int x, int p) {
        dfn[x] = low[x] = cur++;
        stk.push_back(x);
        
        for (auto y : adj[x]) {
            if (y == p) {
                continue;
            }
            if (dfn[y] == -1) {
                E.emplace(x, y);
                dfs(y, x);
                low[x] = std::min(low[x], low[y]);
            } else if (bel[y] == -1 && dfn[y] < dfn[x]) {
                E.emplace(x, y);
                low[x] = std::min(low[x], dfn[y]);
            }
        }
        
        if (dfn[x] == low[x]) {
            int y;
            do {
                y = stk.back();
                bel[y] = cnt;
                stk.pop_back();
            } while (y != x);
            cnt++;
        }
    }
    
    std::vector<int> work() {
        dfs(0, -1);
        return bel;
    }
    
    struct Graph {
        int n;
        std::vector<std::pair<int, int>> edges;
        std::vector<int> siz;
        std::vector<int> cnte;
    };
    Graph compress() {
        Graph g;
        g.n = cnt;
        g.siz.resize(cnt);
        g.cnte.resize(cnt);
        for (int i = 0; i < n; i++) {
            g.siz[bel[i]]++;
            for (auto j : adj[i]) {
                if (bel[i] < bel[j]) {
                    g.edges.emplace_back(bel[i], bel[j]);
                } else if (i < j) {
                    g.cnte[bel[i]]++;
                }
            }
        }
        return g;
    }
};
\end{lstlisting}


\subsection{MaxFlow}
\begin{lstlisting}
template<class T>
struct MaxFlow {
    struct _Edge {
        int to;
        T cap;
        _Edge(int to, T cap) : to(to), cap(cap) {}
    };
    int n;
    std::vector<_Edge> e;
    std::vector<std::vector<int>> g;
    std::vector<int> cur, h;
    MaxFlow() {}
    MaxFlow(int n) {
        init(n);
    }
    void init(int n) {
        this->n = n;
        e.clear();
        g.assign(n, {});
        cur.resize(n);
        h.resize(n);
    }
    bool bfs(int s, int t) {
        h.assign(n, -1);
        std::queue<int> que;
        h[s] = 0;
        que.push(s);
        while (!que.empty()) {
            const int u = que.front();
            que.pop();
            for (int i : g[u]) {
                auto [v, c] = e[i];
                if (c > 0 && h[v] == -1) {
                    h[v] = h[u] + 1;
                    if (v == t) {
                        return true;
                    }
                    que.push(v);
                }
            }
        }
        return false;
    }
    T dfs(int u, int t, T f) {
        if (u == t) {
            return f;
        }
        auto r = f;
        for (int &i = cur[u]; i < int(g[u].size()); ++i) {
            const int j = g[u][i];
            auto [v, c] = e[j];
            if (c > 0 && h[v] == h[u] + 1) {
                auto a = dfs(v, t, std::min(r, c));
                e[j].cap -= a;
                e[j ^ 1].cap += a;
                r -= a;
                if (r == 0) {
                    return f;
                }
            }
        }
        return f - r;
    }
    void addEdge(int u, int v, T c) {
        g[u].push_back(e.size());
        e.emplace_back(v, c);
        g[v].push_back(e.size());
        e.emplace_back(u, 0);
    }
    T flow(int s, int t) {
        T ans = 0;
        while (bfs(s, t)) {
            cur.assign(n, 0);
            ans += dfs(s, t, std::numeric_limits<T>::max());
        }
        return ans;
    }
    std::vector<bool> minCut() {
        std::vector<bool> c(n);
        for (int i = 0; i < n; i++) {
            c[i] = (h[i] != -1);
        }
        return c;
    }

    
    struct Edge {
        int from;
        int to;
        T cap;
        T flow;
    };
    std::vector<Edge> edges() {
        std::vector<Edge> a;
        for (int i = 0; i < e.size(); i += 2) {
            Edge x;
            x.from = e[i + 1].to;
            x.to = e[i].to;
            x.cap = e[i].cap + e[i + 1].cap;
            x.flow = e[i + 1].cap;
            a.push_back(x);
        }
        return a;
    }
};
\end{lstlisting}

\subsection{Minflow}
\begin{lstlisting}
template<class T>
struct MinCostFlow {
    struct _Edge {
        int to;
        T cap;
        T cost;
        _Edge(int to_, T cap_, T cost_) : to(to_), cap(cap_), cost(cost_) {}
    };
    int n;
    std::vector<_Edge> e;
    std::vector<std::vector<int>> g;
    std::vector<T> h, dis;
    std::vector<int> pre;
    bool dijkstra(int s, int t) {
        dis.assign(n, std::numeric_limits<T>::max());
        pre.assign(n, -1);
        std::priority_queue<std::pair<T, int>, std::vector<std::pair<T, int>>, std::greater<std::pair<T, int>>> que;
        dis[s] = 0;
        que.emplace(0, s);
        while (!que.empty()) {
            T d = que.top().first;
            int u = que.top().second;
            que.pop();
            if (dis[u] != d) {
                continue;
            }
            for (int i : g[u]) {
                int v = e[i].to;
                T cap = e[i].cap;
                T cost = e[i].cost;
                if (cap > 0 && dis[v] > d + h[u] - h[v] + cost) {
                    dis[v] = d + h[u] - h[v] + cost;
                    pre[v] = i;
                    que.emplace(dis[v], v);
                }
            }
        }
        return dis[t] != std::numeric_limits<T>::max();
    }
    MinCostFlow() {}
    MinCostFlow(int n_) {
        init(n_);
    }
    void init(int n_) {
        n = n_;
        e.clear();
        g.assign(n, {});
    }
    void addEdge(int u, int v, T cap, T cost) {
        g[u].push_back(e.size());
        e.emplace_back(v, cap, cost);
        g[v].push_back(e.size());
        e.emplace_back(u, 0, -cost);
    }
    std::pair<T, T> flow(int s, int t) {
        T flow = 0;
        T cost = 0;
        h.assign(n, 0);
        while (dijkstra(s, t)) {
            for (int i = 0; i < n; ++i) {
                h[i] += dis[i];
            }
            T aug = std::numeric_limits<int>::max();
            for (int i = t; i != s; i = e[pre[i] ^ 1].to) {
                aug = std::min(aug, e[pre[i]].cap);
            }
            for (int i = t; i != s; i = e[pre[i] ^ 1].to) {
                e[pre[i]].cap -= aug;
                e[pre[i] ^ 1].cap += aug;
            }
            flow += aug;
            cost += aug * h[t];
        }
        return std::make_pair(flow, cost);
    }
    struct Edge {
        int from;
        int to;
        T cap;
        T cost;
        T flow;
    };
    std::vector<Edge> edges() {
        std::vector<Edge> a;
        for (int i = 0; i < e.size(); i += 2) {
            Edge x;
            x.from = e[i + 1].to;
            x.to = e[i].to;
            x.cap = e[i].cap + e[i + 1].cap;
            x.cost = e[i].cost;
            x.flow = e[i + 1].cap;
            a.push_back(x);
        }
        return a;
    }
};
\end{lstlisting}



\subsection{可行流/最大流}
\begin{lstlisting}
struct MCFGraph {
    struct Edge {
        int v, c, f;
        Edge(int v, int c, int f) : v(v), c(c), f(f) {}
    };
    const int n;
    std::vector<Edge> e;
    std::vector<std::vector<int>> g;
    std::vector<i64> h, dis;
    std::vector<int> pre;
    bool dijkstra(int s, int t) {
        dis.assign(n, std::numeric_limits<i64>::max());
        pre.assign(n, -1);
        std::priority_queue<std::pair<i64, int>, std::vector<std::pair<i64, int>>, std::greater<std::pair<i64, int>>> que;
        dis[s] = 0;
        que.emplace(0, s);
        while (!que.empty()) {
            i64 d = que.top().first;
            int u = que.top().second;
            que.pop();
            if (dis[u] < d) continue;
            for (int i : g[u]) {
                int v = e[i].v;
                int c = e[i].c;
                int f = e[i].f;
                if (c > 0 && dis[v] > d + h[u] - h[v] + f) {
                    dis[v] = d + h[u] - h[v] + f;
                    pre[v] = i;
                    que.emplace(dis[v], v);
                }
            }
        }
        return dis[t] != std::numeric_limits<i64>::max();
    }
    MCFGraph(int n) : n(n), g(n) {}
    //可行流
    void addEdge(int u, int v, int c, int f) {
        if (f < 0) {
            g[u].push_back(e.size());
            e.emplace_back(v, 0, f);
            g[v].push_back(e.size());
            e.emplace_back(u, c, -f);
        } else {
            g[u].push_back(e.size());
            e.emplace_back(v, c, f);
            g[v].push_back(e.size());
            e.emplace_back(u, 0, -f);
        }
    }
    //最大流
    /*void addEdge(int u, int v, int c, int f) {
    g[u].push_back(e.size());
    e.emplace_back(v, c, f);
    g[v].push_back(e.size());
    e.emplace_back(u, 0, -f);
    }*/
    std::pair<int, i64> flow(int s, int t) {
        int flow = 0;
        i64 cost = 0;
        h.assign(n, 0);
        while (dijkstra(s, t)) {
            for (int i = 0; i < n; ++i) h[i] += dis[i];
            int aug = std::numeric_limits<int>::max();
            for (int i = t; i != s; i = e[pre[i] ^ 1].v) aug = std::min(aug, e[pre[i]].c);
            for (int i = t; i != s; i = e[pre[i] ^ 1].v) {
                e[pre[i]].c -= aug;
                e[pre[i] ^ 1].c += aug;
            }
            flow += aug;
            cost += i64(aug) * h[t];
        }
        return std::make_pair(flow, cost);
    }
};
\end{lstlisting}

\subsection{TwoSat}
\begin{lstlisting}
struct TwoSat {
    int n;
    std::vector<std::vector<int>> e;
    std::vector<bool> ans;
    TwoSat(int n) : n(n), e(2 * n), ans(n) {}
    void addClause(int u, bool f, int v, bool g) {
        e[2 * u + !f].push_back(2 * v + g);
        e[2 * v + !g].push_back(2 * u + f);
    }
    bool satisfiable() {
        std::vector<int> id(2 * n, -1), dfn(2 * n, -1), low(2 * n, -1);
        std::vector<int> stk;
        int now = 0, cnt = 0;
        std::function<void(int)> tarjan = [&](int u) {
            stk.push_back(u);
            dfn[u] = low[u] = now++;
            for (auto v : e[u]) {
                if (dfn[v] == -1) {
                    tarjan(v);
                    low[u] = std::min(low[u], low[v]);
                } else if (id[v] == -1) {
                    low[u] = std::min(low[u], dfn[v]);
                }
            }
            if (dfn[u] == low[u]) {
                int v;
                do {
                    v = stk.back();
                    stk.pop_back();
                    id[v] = cnt;
                } while (v != u);
                ++cnt;
            }
        };
        for (int i = 0; i < 2 * n; ++i) if (dfn[i] == -1) tarjan(i);
        for (int i = 0; i < n; ++i) {
            if (id[2 * i] == id[2 * i + 1]) return false;
            ans[i] = id[2 * i] > id[2 * i + 1];
        }
        return true;
    }
    std::vector<bool> answer() { return ans; }
};
\end{lstlisting}

