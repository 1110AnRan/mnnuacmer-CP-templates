\subsubsection{vector}
\begin{center}
\begin{tabular}{| c | c | c |}
\hline
\multicolumn{3}{| c |}{C++ vector 关键特性总结} \\
\hline
常用函数/方法 & 时间复杂度 & 说明 \\
\hline
operator[] & \multirow{7}{*}{O(1)} & 随机访问,无边界检查 \\
front() & & 访问首元素 \\
back() & & 访问尾元素 \\
size() & & 获取元素数量 \\
empty() & & 判空 \\
pop\_back() & & 删除尾部元素 \\
push\_back(const T\& value) & & 尾部插入 \\
\hline
resize(n) & O(|n-size|) & 多出元素默认初始化,超出部分删除 \\
\hline
insert(size\_t pos, const T\& value) & \multirow{4}{*}{O(n)} & 在位置 pos 插入元素 \\
erase(size\_t pos) & & 删除位置 pos 元素 \\
clear() & & 清空元素 \\
assign(size\_t n, const T\& value) & & 覆盖内容为 n 个 val \\
\hline
\end{tabular}
\end{center}
\begin{lstlisting}
// 1. 初始化vector的多种方式
std::vector<int> vec1;                  // 空vector
std::vector<int> vec2(5, 10);           // 5个元素,每个都是10
std::vector<int> vec3 = {1, 3, 5, 7, 9}; // 初始化列表 (C++11)
std::vector<int> vec4(vec3);            // 拷贝构造

// 2. 添加元素
vec1.push_back(100);      // 末尾添加元素
vec1.push_back(200);
vec1.insert(vec1.begin(), 50); // 在开头插入元素
vec1.emplace_back(300);  // 高效末尾添加 (C++11)

// 3. 访问元素
std::cout << "第一个元素: " << vec1.front() << "\n";  // 50
std::cout << "最后一个元素: " << vec1.back() << "\n"; // 300
std::cout << "索引1的元素: " << vec1[1] << "\n";      // 100
std::cout << "安全访问索引2: " << vec1.at(2) << "\n"; // 200

// 4. 遍历vector
std::cout << "\n遍历vec1: ";
for (int num : vec1) {  // 范围for循环 (C++11)
    std::cout << num << " ";
}

// 迭代器遍历
std::cout << "\n迭代器遍历: ";
for (auto it = vec1.begin(); it != vec1.end(); ++it) {
    std::cout << *it << " ";
}

// 5. 修改元素
vec1[0] = 55;          // 通过索引修改
vec1.front() = 50;      // 恢复原值

// 6. 容量操作
std::cout << "\n\n大小: " << vec1.size();

// 7. 删除元素
vec1.pop_back();        // 删除末尾元素
vec1.erase(vec1.begin()); // 删除第一个元素
// vec1.clear();        // 清空所有元素

// 8. 排序
std::vector<int> nums = {9, 2, 7, 4, 5};
std::sort(nums.begin(), nums.end());

std::cout << "\n\n排序后: ";
for (int n : nums) std::cout << n << " ";

// 9. 二维vector
std::vector<std::vector<int>> matrix = {
    {1, 2, 3},
    {4, 5},
    {6, 7, 8, 9}
};

std::cout << "\n\n二维vector: \n";
for (const auto& row : matrix) {
    for (int val : row) {
        std::cout << val << " ";
    }
    std::cout << "\n";
}

// 10. 交换
std::vector<int> a = {1, 2, 3};
std::vector<int> b = {4, 5};
a.swap(b);  // 交换内容

// 11. 赋值
nums.assign(5, 5);
// 5 5 5 5 5
for(auto x : nums) {
    std::cout << x << " ";
}
std::cout << std::endl;

// 12. resize
nums.resize(10);
// 5 5 5 5 5 0 0 0 0 0
for(auto x : nums) {
    std::cout << x << " ";
}
std::cout << std::endl;

nums.resize(3);
// 5 5 5
for(auto x : nums) {
    std::cout << x << " ";
}
std::cout << std::endl;
\end{lstlisting}