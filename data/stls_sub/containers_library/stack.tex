\subsubsection{stack}
\begin{center}
\begin{tabular}{| c | c | c |}
\hline
\multicolumn{3}{| c |}{C++ stack 关键特性总结} \\
\hline
常用函数/方法 & 时间复杂度 & 说明 \\
\hline
push(const T\& val) & \multirow{5}{*}{O(1)} & 在栈顶插入元素 \\
pop() & & 移除栈顶元素 \\
top() & & 访问栈顶元素 \\
empty() & & 检查栈是否为空 \\
size() & & 返回栈中元素数量 \\
\hline
\end{tabular}
\end{center}

\begin{lstlisting}
// 1. 创建stack的不同方式
std::stack<int> s1; // 默认使用deque作为底层容器

// 2. 基本操作
// 压入元素
s1.push(10);
s1.push(20);
s1.push(30);
s1.emplace(40); // C++11 直接构造元素,避免复制

// 访问栈顶元素
// 40
std::cout << "栈顶元素: " << s1.top() << std::endl; // 40

// 弹出元素
s1.pop(); // 移除40
// 30
std::cout << "弹出后栈顶元素: " << s1.top() << std::endl; // 30

// 3. 容量查询
// 否
std::cout << "栈是否为空: " << (s1.empty() ? "是" : "否") << std::endl;
// 3
std::cout << "栈的大小: " << s1.size() << std::endl;

// 4. 栈的交换
std::stack<int> s2;
s2.push(100);
s2.push(200);

std::cout << "\n交换前:";
// 3
std::cout << "\ns1 大小: " << s1.size();
// 2
std::cout << "\ns2 大小: " << s2.size();

s1.swap(s2); // C++03 方式
// std::swap(s1, s2); // C++11 也可用此方式

std::cout << "\n\n交换后:";
// 2
std::cout << "\ns1 大小: " << s1.size();
// 3
std::cout << "\ns2 大小: " << s2.size();
// 200
std::cout << "\ns1 栈顶: " << s1.top() << std::endl; // 200

// 5. 清空栈
// 200 100
std::cout << "\n清空栈: ";
while (!s1.empty()) {
    std::cout << s1.top() << " ";
    s1.pop();
}
// 0
std::cout << "\n清空后大小: " << s1.size() << std::endl;
// 6. 注意事项
std::stack<int> s3;
// 危险操作:空栈时访问top()会导致未定义行为
// std::cout << s6.top(); // 崩溃!

// 安全访问方式
if (!s3.empty()) {
    std::cout << "栈顶: " << s3.top() << std::endl;
}
else {
    std::cout << "\n警告:尝试访问空栈!" << std::endl;
}
\end{lstlisting}