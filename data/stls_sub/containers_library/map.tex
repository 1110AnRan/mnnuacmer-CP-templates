\subsubsection{map}
\begin{center}
\begin{tabular}{| c | c | c |}
\hline
\multicolumn{3}{| c |}{C++ map 关键特性总结} \\
\hline
常用函数/方法 & 时间复杂度 & 说明 \\
\hline
insert({const K\& key, const V\& value}) & \multirow{9}{*}{O(log n)} & 键存在时插入失败 \\
operator[](const K\& key) & & 键不存在时自动插入默认值 \\
erase(const K\& key) & & 删除指定键 \\
find(const K\& key) & & 返回迭代器或 end() \\
equal\_range(const K\& key) & & 返回单元素的迭代器范围 \\
lower\_bound(const K\& key) & & 首个 $>=$ key 的元素 \\
upper\_bound(const K\& key) & & 首个 $>$ key 的元素 \\
extract(const K\& key) & & 安全转移元素 \\
count(const K\& key) & & 键重复次数 \\
\hline
\end{tabular}
\end{center}
\begin{lstlisting}
std::map<int, int> mp;
mp[1] = 2;
mp[2] = 3;
mp[3] = 4;

// 2
std::cout << mp[1] << std::endl;
// 0
std::cout << mp[4] << std::endl;

mp.erase(4);
// 4 is not in mp
if(mp.count(4)) {
    std::cout << "4 is in mp" << std::endl;
} else {
    std::cout << "4 is not in mp" << std::endl;
}
auto it = mp.find(3);
// 3 is in mp
// 3->4
if(it != mp.end()) {
    std::cout << "3 is in mp" << std::endl;
    std::cout << 3 << "->" << it->second << std::endl;
} else {
    std::cout << "3 is not in mp" << std::endl;
}

it = mp.lower_bound(2);
// ...


mp.extract(2);
// 1->2
// 3->4
for(auto [x, y] : mp) {
    std::cout << x << "->" << y << std::endl;
}
\end{lstlisting}