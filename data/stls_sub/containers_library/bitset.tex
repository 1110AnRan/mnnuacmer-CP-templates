\subsubsection{bitset}
\begin{center}
\begin{tabular}{| c | c | c |}
\hline
\multicolumn{3}{| c |}{C++ bitset 关键特性总结} \\
\hline
常用函数/方法 & 时间复杂度 & 说明 \\
\hline
set() & \multirow{3}{*}{O(n)} & 设置所有位为1 \\
reset() & & 设置所有位为0 \\
flip() & & 翻转所有位 \\
\hline
set(size\_t pos) & \multirow{3}{*}{O(1)} & 设置pos位为1 \\ 
reset(size\_t pos) & & 设置pos位为0 \\
flip(size\_t pos) & & 翻转第pos位 \\
\hline
any() & \multirow{3}{*}{O(n) 或 O(1)} & 是否有1 \\
none() & & 是否全0 \\
all() & & 是否全1 \\
\hline
operator[] & O(1) & 随机访问 \\
\hline
operator\& & \multirow{6}{*}{O(n)} & 与运算 \\
operator| & & 或运算 \\
operator\textasciicircum & & 异或运算 \\
operator\textasciitilde & & 取反 \\
operator<< & & 左移 \\
operator>> & & 右移 \\
\hline
\end{tabular}
\end{center}
\begin{lstlisting}
// 00000000
std::bitset<8> b1;
// 00101010(42的二进制)
std::bitset<8> b2(42);
// 字符串初始化
std::bitset<8> b3("11001010");

// 2. 位操作
// 第0位置1: 00000001
b1.set(0);
// 第0位置0: 00000000
b1.reset(0);
// 第2位翻转: 00000100
b1.flip(2);
// 所有位置1: 11111111
b1.set();
// 所有位置0: 00000000
b1.reset();

// 3. 访问与查询
// 输出第1位(0-based,输出1)
std::cout << b2[1] << std::endl;
// 安全访问(越界抛异常)
std::cout << b2.test(1) << std::endl;
// 统计1的个数(42 → 3)
std::cout << b2.count() << std::endl;
// 是否有1(true)
std::cout << b2.any() << std::endl;
// 是否全0(false)
std::cout << b2.none() << std::endl;
// 是否全1(false)
std::cout << b2.all() << std::endl;

// 4. 位运算
// b1 : 00000000
// b2 : 00101010
// b3 : 11001010
// 00000000
std::cout << (b1 & b3) << std::endl;
// 00101010
std::cout << (b1 | b2) << std::endl;
// 11100000
std::cout << (b2 ^ b3) << std::endl;
// 10101000
std::cout << (b2 << 2) << std::endl;
\end{lstlisting}