\subsubsection{priority\_queue}
\begin{center}
\begin{tabular}{ | c | c | c |}
\hline
\multicolumn{3}{| c |}{C++ priority\_queue 关键特性总结} \\
\hline
常用函数/方法 & 复杂度 & 说明 \\
\hline
push(const T\& value) & \multirow{2}{*}{O(log n)} & 添加新元素 \\
pop() & & 移除堆顶元素 \\
\hline
top() & \multirow{3}{*}{O(1)} & 获取最高优先级元素 \\
empty() & & 判断队列是否为空 \\
size() & & 返回元素数量 \\
\hline
\end{tabular}
\end{center}

\begin{lstlisting}
// 初始化
// 默认大根堆(最大元素在顶部)
std::priority_queue<int> p1;
// 小根据(最小元素在顶部)
std::priority_queue<int, std::vector<int>, std::greater<int>> p2;

{
    struct Node {
        int id, dist;

        // 重载 < 运算符(大根堆)
        bool operator<(const Node& other) const {
            return dist < other.dist; // dist 大的优先级高
        }
        // 重载 > 运算符(小根堆)
        bool operator>(const Node& other) const {
            return dist > other.dist; // dist 小的优先级高
        }
    };
    // 大根堆
    std::priority_queue<Node> p1;
    // 小根堆
    std::priority_queue<Node, std::vector<Node>, std::greater<Node>> p2;

    // 按距离升序(小根堆)
    struct CompareDist {
        bool operator()(const Node& a, const Node& b) {
            return a.dist > b.dist;
        }
    };

    // 自定义排序
    std::priority_queue<Node, std::vector<Node>, CompareDist> p3;
    
    auto cmp = [](const Node& a, const Node& b) {
        return a.dist > b.dist; // 小根堆
    };
    // lambda 表达式
    std::priority_queue<Node, std::vector<Node>, decltype(cmp)> p4(cmp);
}

p1.push(10);
p1.push(20);
p1.push(30);

p2.push(10);
p2.push(20);
p2.push(30);



// 30 20 10
while(!p1.empty()) {
    std::cout << p1.top() << std::endl;
    p1.pop();
}


// 10 20 30
while(!p2.empty()) {
    std::cout << p2.top() << std::endl;
    p2.pop();
}
\end{lstlisting}