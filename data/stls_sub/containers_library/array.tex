\subsubsection{array}
\begin{center}
\begin{tabular}{| c | c | c |}
\hline
\multicolumn{3}{| c |}{C++ array 关键特性总结} \\
\hline
常用函数/方法 & 时间复杂度 & 说明 \\
\hline
operator[] & \multirow{5}{*}{O(1)} & 随机访问(无边界检查) \\
front() & & 访问首元素 \\
back() & & 访问尾元素 \\
size() & & 返回元素数量(编译时确定) \\
empty() & & 检查是否为空(编译时确定)\\
\hline
fill(const T\& value) & O(n) & 填充所有元素为指定值 \\
\hline
\end{tabular}
\end{center}
\begin{lstlisting}
// 默认构造(元素未初始化!)
std::array<int, 5> arr1; 

// 列表初始化(C++11)
std::array<int, 5> arr2 = {1, 2, 3, 4, 5};

// 统一初始化(C++11)
std::array<std::string, 3> arr3{"Apple", "Banana", "Cherry"};

// 拷贝构造
std::array<int, 5> arr4 = arr2;

// 聚合初始化(C++17)
std::array arr5 = {1.1, 2.2, 3.3}; // 自动推导类型和大小

// 下标访问
std::cout << "索引1:" << arr2[1] << std::endl;

// 首尾元素访问
std::cout << "首元素: " << arr2.front() << std::endl; // 1
std::cout << "尾元素: " << arr2.back() << std::endl;  // 5

// 容量操作
std::cout << "大小:" << arr2.size() << std::endl;

// 范围for循环
std::cout << "范围遍历:";
for(auto x : arr2) {
    std::cout << x << " "; 
}
std::cout << std::endl;

// arr1:[10, 10, 10, 10, 10]
arr1.fill(10);

// 使用STL算法修改
std::sort(arr2.begin(), arr2.end()); // 升序排序
std::reverse(arr2.begin(), arr2.end()); // 反转

// 结构化绑定
std::array<int, 3> point = {10, 20, 30};
auto [x, y, z] = point;
// 10 20 30
std::cout << "坐标: (" << x << ", " << y << ", " << z << ")";
\end{lstlisting}