\subsubsection{queue}
\begin{center}
\begin{tabular}{| c | c | c |}
\hline
\multicolumn{3}{| c |}{C++ queue 关键特性总结} \\
\hline
常用函数/方法 & 时间复杂度 & 说明 \\
\hline
push(const T\& value) & \multirow{6}{*}{O(1)} & 在队尾插入元素 \\
pop() & & 移除队头元素 \\
front() & & 访问队头元素 \\
back() & & 访问队尾元素 \\
empty() & & 检查队列是否为空 \\
size() & & 返回队列中元素数量 \\
\hline
\end{tabular}
\end{center}
\begin{lstlisting}
// 1. 创建queue的不同方式
std::queue<int> q1; // 默认使用deque作为底层容器

// 2. 基本操作
// 入队操作
q1.push(10);
q1.push(20);
q1.push(30);
q1.emplace(40); // C++11 直接构造元素

// 访问队首和队尾元素
// 10
std::cout << "队首元素: " << q1.front() << std::endl;
// 40
std::cout << "队尾元素: " << q1.back() << std::endl;

// 出队操作
q1.pop(); // 移除10
// 20
std::cout << "\n出队后队首元素: " << q1.front() << std::endl;
// 40
std::cout << "当前队尾元素: " << q1.back() << std::endl;

// 3. 容量查询
std::cout << "\n队列是否为空: " << (q1.empty() ? "是" : "否") << std::endl;
std::cout << "队列的大小: " << q1.size() << std::endl;

// 4. 队列的交换
std::queue<int> q2;
q2.push(100);
q2.push(200);

std::cout << "\n交换前:";
std::cout << "\nq1 大小: " << q1.size();
std::cout << "\nq4 大小: " << q2.size();

q1.swap(q2); // 交换内容

std::cout << "\n\n交换后:";
std::cout << "\nq1 大小: " << q1.size();
std::cout << "\nq4 大小: " << q2.size();
// 100
std::cout << "\nq1 队首: " << q1.front();
// 200
std::cout << "\nq1 队尾: " << q1.back();

// 5. 清空队列
std::cout << "\n\n清空q4: ";
while (!q2.empty()) {
    std::cout << q2.front() << " ";
    q2.pop();
}
std::cout << "\n清空后大小: " << q2.size() << std::endl;

// 6. 注意事项
std::queue<int> emptyQueue;

// 危险操作:空队列时访问front()或back()会导致未定义行为
// std::cout << emptyQueue.front(); // 崩溃!

// 安全访问方式
if (!emptyQueue.empty()) {
    std::cout << "\n队首元素: " << emptyQueue.front() << std::endl;
} else {
    std::cout << "\n\n警告:尝试访问空队列!" << std::endl;
}
\end{lstlisting}