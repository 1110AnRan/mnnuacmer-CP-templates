\subsection{数学定理}

\begin{theorem}[欧几里得]
若 $a, b$ 为整数,$b \ne 0$,则有 $\gcd(a, b) = \gcd(b, a \bmod b)$。
\end{theorem}
\begin{lstlisting}
// 求最大公约数
int gcd(int a, int b) {
    return b == 0 ? a : gcd(b, a % b);
}
\end{lstlisting}

\begin{theorem}[扩展欧几里得]
设 $a, b \in \mathbb{Z}$ 且不全为零,则存在整数 $x, y$ 使得:
\[
ax + by = \gcd(a, b)
\]
\end{theorem}
\begin{lstlisting}
// 求一组解 (x, y) 满足 ax + by = gcd(a, b)
int exgcd(int a, int b, int &x, int &y) {
    if (!b) { x = 1; y = 0; return a; }
    int d = exgcd(b, a % b, y, x);
    y -= (a / b) * x;
    return d;
}
\end{lstlisting}


\begin{theorem}[模逆元]
若 $\gcd(a, m) = 1$,则存在整数 $x$,使得:
\[
ax \equiv 1 \pmod{m}
\]
称 $x$ 为 $a$ 在模 $m$ 下的乘法逆元
\end{theorem}
\begin{lstlisting}
// 计算 a 在 mod m 下的逆元(gcd(a, m) == 1)
int modInverse(int a, int m) {
    int x, y;
    int d = exgcd(a, m, x, y);
    if (d != 1) return -1;
    return (x % m + m) % m;
}
\end{lstlisting}

\begin{theorem}[线性不定方程求解]
方程 $ax + by = c$ 有整数解当且仅当 $\gcd(a, b) \mid c$。 \\
如果 $ax + by = d \quad d$ 为 $\gcd(a, b)$,其中一个特解是 $(x_0, y_0)$ \\
则通解为:
\[
x = x_0 + \frac{b}{d} \ast n \qquad y = y_0 - \frac{a}{d} \ast n \quad n \in \mathbb{Z}
\]
如果 $ax + by = c \quad c$ 为 $d$ 的整数倍,根据上面的特解,可以得到该等式的一个特解 $(x_0', y_0')$ \\
其中 $x_0' = x_0 \ast \frac{c}{d} \quad y_0' = y_0 \ast \frac{c}{d}$ \\
则通解为:
\[
x = x_0' + \frac{b}{d} \ast n \quad y = y_0' - \frac{a}{d} \ast n \quad n \in \mathbb{Z}
\]
\end{theorem}
\begin{lstlisting}
// 解线性不定方程,返回一组整数解
bool solveDiophantine(int a, int b, int c, int &x, int &y) {
    int d = exgcd(a, b, x, y);
    if (c % d != 0) return false;
    int k = c / d;
    x *= k;
    y *= k;
    return true;
}
\end{lstlisting}

\begin{theorem}[解模线性方程]
模线性方程 $ax \equiv b \pmod{m}$ 有解当且仅当 $gcd(a, m) \mid b$。
\end{theorem}

\begin{lstlisting}
// 解 ax ≡ b mod m 的最小正整数解
int modLinearSolve(int a, int b, int m) {
    int x, y;
    int d = exgcd(a, m, x, y);
    if (b % d != 0) return -1;
    x = x * (b / d);
    return (x % (m / d) + (m / d)) % (m / d);
}
\end{lstlisting}

\begin{theorem}[中国剩余定理]
若 $\{m_i\}$ 两两互质,则同余方程组
\[
x \equiv a_i \pmod{m_i}, \quad i =  1,2,...,n
\]
有唯一解 $x \mod M$,其中 $M = \prod m_i$。
\end{theorem}
\begin{lstlisting}
// CRT 模数互质情况,返回最小非负解 x
using long long = i64;

i64 CRT(const std::vector<int>& a, const std::vector<int>& m) {
    i64 M = 1;
    for (int mi : m) M *= mi;

    i64 res = 0;
    for (int i = 0; i < m.size(); ++i) {
        i64 Mi = M / m[i];
        int inv = modInverse(Mi % m[i], m[i]);
        res = (res + 1LL * a[i] * Mi % M * inv % M) % M;
    }
    return (res + M) % M;
}
\end{lstlisting}

\begin{theorem}[扩展中国剩余定理]
对于模数不互质的同余方程组,若
\[
\gcd(m_i, m_j) \mid (a_i - a_j), \quad \forall i, j
\]
则存在整数解 $x$,并可通过逐步合并方式递推构造。
\end{theorem}

\begin{lstlisting}
// exCRT 模数不互质情况,返回最小非负解 x
using long long = i64;

i64 exCRT(const std::vector<i64>& a, const std::vector<i64>& m) {
    i64 x = a[0], mod = m[0];
    for (int i = 1; i < a.size(); ++i) {
        i64 a1 = mod, a2 = m[i];
        i64 b = (a[i] - x % a2 + a2) % a2;

        i64 s, t;
        i64 d = exgcd(a1, a2, s, t);
        if (b % d != 0) return -1;

        i64 k = b / d;
        s = (s % a2 + a2) % a2;
        i64 tmp = (k * s) % (a2 / d);
        x = x + tmp * mod;
        mod = mod / d * a2;
        x = (x % mod + mod) % mod;
    }
    return x;
}
\end{lstlisting}


\begin{definition}[欧拉函数]
对正整数 $n$, $\varphi(n)$ 表示不超过 $n$ 且与 $n$ 互质的正整数个数。
\end{definition}

\begin{lstlisting}
// 单点欧拉函数
int euler(int n) {
    int res = n;
    for (int i = 2; i * i <= n; ++i)
        if (n % i == 0) {
            res = res / i * (i - 1);
            while (n % i == 0) n /= i;
        }
    if (n > 1) res = res / n * (n - 1);
    return res;
}
\end{lstlisting}

\begin{theorem}[鞋带公式]
设有多边形的顶点 $(x_1, y_1), (x_2, y_2), \dots, (x_n, y_n)$,按照顺时针或逆时针顺序排列,闭合多边形的面积 $A$ 为:
\[
A = \frac{1}{2} \left| \sum_{i=1}^{n} (x_i y_{i+1} - x_{i+1} y_i) \right|
\]
其中 $(x_{n+1}, y_{n+1})$ 被认为是 $(x_1, y_1)$。
\end{theorem}

\begin{lstlisting}
// 鞋带公式计算多边形面积
double shoelaceFormula(std::vector<pair<int, int>>& points) {
    int n = points.size();
    double area = 0.0;
    for (int i = 0; i < n; ++i) {
        int j = (i + 1) % n;
        area += (points[i].first * points[j].second - points[i].second * points[j].first);
    }
    return abs(area) / 2.0;
}
\end{lstlisting}

\begin{theorem}[Pick 定理]
设一个简单多边形的顶点均为整数坐标,且该多边形的面积为 $A$,内部的格点数为 $I$,边上的格点数为 $B$,则有:
\[
A = I + \frac{B}{2} - 1
\]
其中 $A$ 表示多边形面积,$I$ 表示多边形内部的格点数,$B$ 表示边上的格点数。
\end{theorem}

\begin{lstlisting}
// Pick 定理的计算:给定顶点和边界点数,计算面积
int pickTheorem(int I, int B) {
    return I + B / 2 - 1;
}
\end{lstlisting}

\begin{theorem}[二项式定理]
对于任意实数 $x$ 和 $y$,以及非负整数 $n$,有:
\[
(x + y)^n = \sum_{k=0}^{n} \binom{n}{k} x^{n-k} y^k
\]
其中 $\binom{n}{k}$ 为二项式系数,表示为:
\[
\binom{n}{k} = \frac{n!}{k!(n-k)!}
\]
\end{theorem}

\begin{theorem}[二项式反演]
二项式反演的四种形式
\begin{align}
    g(n) = \sum_{i = 0}^{n} (-1)^i \binom{n}{i}f(i) \Longleftrightarrow f(n) = \sum_{i = 0}^{n} (-1)^i \binom{n}{i} g(i) \\
    g(n) = \sum_{i = 0}^{n} \binom{n}{i} f(i) \Longleftrightarrow f(n) = \sum_{i = 0}^{n} (-1)^{n - i} \binom{n}{i} g(i) \\
    g(n) = \sum_{i = n}^{N} (-1)^i \binom{i}{n}f(i) \Longleftrightarrow f(n) = \sum_{i = n}^{N} (-1)^i \binom{i}{n} g(i) \\
    g(n) = \sum_{i = n}^{N} \binom{i}{n} f(i) \Longleftrightarrow f(n) = \sum_{i = n}^{N} (-1)^{i - n} \binom{i}{n} g(i)
\end{align}

\end{theorem}

\begin{theorem}[卢卡斯定理]
若 $p$ 为质数,$n$ 和 $k$ 为非负整数,则二项式系数 $\binom{n}{k} \pmod{p}$ 可以通过以下递归关系计算:
\[
\binom{n}{k} \equiv \prod_{i=0}^{m} \binom{n_i}{k_i} \pmod{p}
\]
其中,$n_i$ 和 $k_i$ 是 $n$ 和 $k$ 在基 $p$ 下的每一位的数值。
\end{theorem}

\begin{theorem}[卡特兰数]
卡特兰公式:
\begin{align}
    f(n) = \binom{2n}{n} - \binom{2n}{n - 1} \tag{1}\\
    f(n) = \frac{\binom{2n}{n}}{(n + 1)}  \tag{2}\\
    f(n) = f(n - 1) \ast \frac{4n - 2}{n + 1}  \tag{3}\\
    f(n) = \sum_{i = 0}^{n - 1} f(i) \ast f(n - 1 - i)\tag{4}
\end{align}
\end{theorem}