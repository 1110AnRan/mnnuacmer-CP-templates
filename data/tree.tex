\section{树论专题模板}
\subsection{BlockCutTree}
\begin{lstlisting}
struct BlockCutTree {
    int n;
    std::vector<std::vector<int>> adj;
    std::vector<int> dfn, low, stk;
    int cnt, cur;
    std::vector<std::pair<int, int>> edges;
    
    BlockCutTree() {}
    BlockCutTree(int n) {
        init(n);
    }
    
    void init(int n) {
        this->n = n;
        adj.assign(n, {});
        dfn.assign(n, -1);
        low.resize(n);
        stk.clear();
        cnt = cur = 0;
        edges.clear();
    }
    
    void addEdge(int u, int v) {
        adj[u].push_back(v);
        adj[v].push_back(u);
    }
    
    void dfs(int x) {
        stk.push_back(x);
        dfn[x] = low[x] = cur++;
        
        for (auto y : adj[x]) {
            if (dfn[y] == -1) {
                dfs(y);
                low[x] = std::min(low[x], low[y]);
                if (low[y] == dfn[x]) {
                    int v;
                    do {
                        v = stk.back();
                        stk.pop_back();
                        edges.emplace_back(n + cnt, v);
                    } while (v != y);
                    edges.emplace_back(x, n + cnt);
                    cnt++;
                }
            } else {
                low[x] = std::min(low[x], dfn[y]);
            }
        }
    }
    
    std::pair<int, std::vector<std::pair<int, int>>> work() {
        for (int i = 0; i < n; i++) {
            if (dfn[i] == -1) {
                stk.clear();
                dfs(i);
            }
        }
        return {cnt, edges};
    }
};
\end{lstlisting}

\subsection{树的重心}
\begin{lstlisting}
struct CodeOfTree {
    int n, cur;
    std::vector<std::vector<int>> adj;
    std::vector<int> siz, dep, p, in, ord;
    CodeOfTree() = default;

    CodeOfTree(int _n) {
        init(_n);
    }

    void init(int _n) {
        this->n = _n;
        cur = 0;
        adj.assign(n, {});
        siz.resize(n);
        dep.resize(n);
        p.resize(n);
        in.resize(n);
        ord.resize(n);
    }

    void addEdge(int u, int v) {
        adj[u].push_back(v);
        adj[v].push_back(u);
    }

    void dfs(int u) {
        siz[u] = 1;
        in[u] = cur++;
        ord[in[u]] = u;
        for(auto v : adj[u]) {
            if(v == p[u]) {
                continue;
            }
            p[v] = u;
            dep[v] = dep[u] + 1;
            dfs(v);
            siz[u] += siz[v];
        }
    }

    int find(int u) {
        for(auto v : adj[u]) {
            if(v == p[u] || 2 * siz[v] <= n) {
                continue;
            }
            return find(v);
        }
        return u;
    }

    void work() {
        p[0] = -1;
        dfs(0);
        int rt = find(0);
        dep[rt] = 0;
        p[rt] = -1;
        cur = 0;
        dfs(rt);
    }
};
\end{lstlisting}

\subsection{树的直径}
\begin{lstlisting}
struct Diameter {  
    int n, start, end;
    std::vector<std::vector<int>> adj;
    std::vector<int> pa;

    Diameter() = default;
    Diameter(int _n) {
        init(_n);
    }

    void init(int _n) {
        this->n = _n;
        start = end = -1;
        adj.assign(n, {});
        pa.resize(n);
        std::iota(pa.begin(), pa.end(), 0);
    }

    void addEdge(int u, int v) {
        adj[u].push_back(v);
        adj[v].push_back(u);
    }

    std::pair<int, int> dfs(int u, int fa, std::vector<bool>& vis) {
        std::pair<int, int> ans{1, u};
        pa[u] = fa;
        vis[u] = true;
        for(auto v : adj[u]) {
            if(v == fa || vis[v]) {
                continue;
            }
            auto pi = dfs(v, u, vis);
            pi.first += 1;
            ans = max(ans, pi);
        }
        return ans;
    }

    std::pair<int, int> work() {
        std::vector<bool> vis(n);
        auto [d1, ss] = dfs(0, -1, vis);
        auto [d2, ee] = dfs(ss, -1, vis);
        start = ss, end = ee;
        return std::make_pair(start, end);
    }

    /* default = end -> start */
    std::vector<int> getPath() {
        std::vector<int> v;
        int cur = end;
        while(cur != -1) {
            v.push_back(cur);
            cur = pa[cur];
        }
        // std::reverse(v.begin(), v.end());
        return v;
    }
};
\end{lstlisting}

\subsection{笛卡尔树}
\begin{lstlisting}
template<class T>
struct DescartesTree {

	std::vector<int> lch, rch, stk;

	DescartesTree() {}

	DescartesTree(const std::vector<T>& v) {
		work(v);
	}

	void work(const std::vector<T>& v) {
		int n = v.size();
		lch.resize(n, n);
		rch.resize(n, n);
		for(int i = 0; i < n; i++) {
			while(!stk.empty() && v[i] < v[stk.back()]) {
				rch[stk.back()] = lch[i];
				lch[i] = stk.back();
				stk.pop_back();
			}
			stk.push_back(i);
		}
		while(stk.size() > 1) {
			int x = stk.back();
			stk.pop_back();
			rch[stk.back()] = x;
		}
	}
};
\end{lstlisting}

\subsection{HLD}
\begin{lstlisting}
struct HLD {
    int n;
    std::vector<int> siz, top, dep, parent, in, out, seq, son;
    std::vector<std::vector<int>> adj;
    int cur;
    
    HLD() {}
    HLD(int n) {
        init(n);
    }
    void init(int n) {
        this->n = n;
        siz.resize(n);
        top.resize(n);
        dep.resize(n);
        parent.resize(n);
        in.resize(n);
        out.resize(n);
        seq.resize(n);
        son.resize(n);
        cur = 0;
        adj.assign(n, {});
    }
    void addEdge(int u, int v) {
        adj[u].push_back(v);
        adj[v].push_back(u);
    }
    void work(int root = 0) {
        top[root] = root;
        dep[root] = 0;
        parent[root] = -1;
        dfs1(root);
        dfs2(root);
    }
    void dfs1(int u) {
        if (parent[u] != -1) {
            adj[u].erase(std::find(adj[u].begin(), adj[u].end(), parent[u]));
        }
        
        siz[u] = 1;
        for (auto &v : adj[u]) {
            parent[v] = u;
            dep[v] = dep[u] + 1;
            dfs1(v);
            siz[u] += siz[v];
            if (siz[v] > siz[adj[u][0]]) {
                son[u] = v;
                std::swap(v, adj[u][0]);
            }
        }
    }
    void dfs2(int u) {
        in[u] = cur++;
        seq[in[u]] = u;
        for (auto v : adj[u]) {
            top[v] = v == adj[u][0] ? top[u] : v;
            dfs2(v);
        }
        out[u] = cur;
    }
    int lca(int u, int v) {
        while (top[u] != top[v]) {
            if (dep[top[u]] > dep[top[v]]) {
                u = parent[top[u]];
            } else {
                v = parent[top[v]];
            }
        }
        return dep[u] < dep[v] ? u : v;
    }
    
    int dist(int u, int v) {
        return dep[u] + dep[v] - 2 * dep[lca(u, v)];
    }
    
    int jump(int u, int k) {
        if (dep[u] < k) {
            return -1;
        }
        
        int d = dep[u] - k;
        
        while (dep[top[u]] > d) {
            u = parent[top[u]];
        }
        
        return seq[in[u] - dep[u] + d];
    }
    
    /* 
     * 判断u是否是v的祖先
    */
    bool isAncester(int u, int v) {
        return in[u] <= in[v] && in[v] < out[u];
    }
    
    int rootedParent(int u, int v) {
        std::swap(u, v);
        if (u == v) {
            return u;
        }
        if (!isAncester(u, v)) {
            return parent[u];
        }
        auto it = std::upper_bound(adj[u].begin(), adj[u].end(), v, [&](int x, int y) {
            return in[x] < in[y];
        }) - 1;
        return *it;
    }
    

    /*
     * 返回在以 v 为根时,节点 u 的子树大小。
    */
    int rootedSize(int u, int v) {
        if (u == v) {
            return n;
        }
        if (!isAncester(v, u)) {
            return siz[v];
        }
        return n - siz[rootedParent(u, v)];
    }
    
    int rootedLca(int a, int b, int c) {
        return lca(a, b) ^ lca(b, c) ^ lca(c, a);
    }

    std::veector<std::pair<int, int>> get_path(int u, int v) {
        std::vector<std::pair<int, int>> v1, v2;
        while(top[u] != top[v]) {
            if(dep[top[u]] > dep[top[v]]) {
                v1.push_back({dfn[u], dfn[top[u]]});
                u = parent[top[u]];
            } else {
                v2.push_back({dfn[top[v], dfn[v]]});
                v = parent[top[v]];
            }
        }
        v1.reserve(v1.size() + v2.size() + 1);
        v1.push_back({dfn[u], dfn[v]});
        reverse(v2.begin(), v2.end());
        for(auto v : v2) {
            v1.push_back(v);
        }
        return v1;
    }
};
\end{lstlisting}

\subsection{Splay}
\begin{lstlisting}
class Splay {
public:
    Splay() {}

    Splay(int n) {
        init(n);
    }

    void init(int n) {
        cnt = head = 0;
        Tree.assign(n + 5, {});
    }

    // add a num to the Tree
    void add(int num) {
        add(head, num);
    }

    // find the rank`s node in the Tree`s inorder
    int find(int rank) {
        return find(head, rank);
    }

    // query the num`s rank in the Tree
    int rank(int num) {
        return rank(head, num);
    }

    // return x-th`s value after sorting
    int index(int x) {
        int i = find(x);
        splay(i, 0);
        return Tree[i].key;
    }

    // return the pre num`s value
    int pre(int num) {
        return pre(head, num);
    }

    // return the post num`s value
    int post(int num) {
        return post(head, num);
    }

    // remove a num from the Tree
    void remove(int num) {
        int kth = rank(num);
        if(kth != rank(num + 1)) {
            int i = find(kth);
            splay(i, 0);
            if(Tree[i].ls == 0) {
                head = Tree[i].rs;
            } else if(Tree[i].rs == 0) {
                head = Tree[i].ls;
            } else {
                int j = find(kth + 1);
                splay(j, i);
                Tree[j].ls = Tree[i].ls;
                Tree[Tree[j].ls].father = j;
                up(j);
                head = j;
            }
            Tree[head].father = 0;
        }
    }

private:

    // Summary all the info about node[i]`s son
    void up(int i) {
        Tree[i].size = Tree[Tree[i].ls].size + Tree[Tree[i].rs].size + 1;
    }

    // check i is the right child of its father
    int lr(int i) {
        return Tree[Tree[i].father].rs == i ? 1 : 0;
    }

    void rotate(int i) {
        int f = Tree[i].father, g = Tree[f].father;
        int soni = lr(i), sonf = lr(f);
        if(soni == 1) {
            Tree[f].rs = Tree[i].ls;
            if(Tree[f].rs != 0) {
                Tree[Tree[f].rs].father = f;
            }
            Tree[i].ls = f;
        } else {
            Tree[f].ls = Tree[i].rs;
            if(Tree[f].ls != 0) {
                Tree[Tree[f].ls].father = f;
            }
            Tree[i].rs = f;
        }
        if(g != 0) {
            if(sonf == 0) {
                Tree[g].ls = i;
            } else {
                Tree[g].rs = i;
            }
        }
        Tree[i].father = g;
        Tree[f].father = i;
        up(f);
        up(i);
    }

    // make node[i] is a child of node[goal]
    void splay(int i, int goal) {
        int f = Tree[i].father, g = Tree[f].father;
        while(f != goal) {
            if(g != goal) {
                if(lr(i) == lr(f)) {
                    rotate(f);
                } else {
                    rotate(i);
                }
            }
            rotate(i);
            f = Tree[i].father;
            g = Tree[f].father;
        }
        if(goal == 0) {
            head = i;
        }
    }

    void add(int i, int num) {
        Tree[++cnt].key = num;
        Tree[cnt].size = 1;
        if(head == 0) {
            head = cnt;
        } else {
            int f = 0, son = 0;
            while(i != 0) {
                f = i;
                if(Tree[i].key <= num) {
                    son = 1;
                    i = Tree[i].rs;
                } else {
                    son = 0;
                    i = Tree[i].ls;
                }
            }
            if(son == 0) {
                Tree[f].ls = cnt;
            } else {
                Tree[f].rs = cnt;
            }
            Tree[cnt].father = f;
            splay(cnt, 0);
        }
    }

    int find(int i, int rank) {
        while(i != 0) {
            if(Tree[Tree[i].ls].size + 1 == rank) {
                return i;
            } else if(Tree[Tree[i].ls].size >= rank) {
                i = Tree[i].ls;
            } else {
                rank -= Tree[Tree[i].ls].size + 1;
                i = Tree[i].rs;
            }
        }
        return 0;
    }

    int rank(int i, int num) {
        int ans = 0, f = 0;
        while(i != 0) {
            f = i;
            if(Tree[i].key >= num) {
                i = Tree[i].ls;
            } else {
                ans += Tree[Tree[i].ls].size + 1;
                i = Tree[i].rs;
            }
        }
        splay(f, 0);
        return ans + 1;
    }

    int pre(int i, int num) {
        int last = head;
        int ans = std::numeric_limits<int>::min();
        while(i != 0) {
            last = i;
            if(Tree[i].key < num) {
                ans = std::max(ans, Tree[i].key);
                i = Tree[i].rs;
            } else {
                i = Tree[i].ls;
            }
        }
        splay(last, 0);
        return ans;
    }

    int post(int i, int num) {
        int last = head;
        int ans = std::numeric_limits<int>::max();
        while(i != 0) {
            last = i;
            if(Tree[i].key > num) {
                ans = std::min(ans, Tree[i].key);
                i = Tree[i].ls;
            } else {
                i = Tree[i].rs;
            }
        }
        splay(last, 0);
        return ans;
    }

    struct Node {
        int ls, rs, size, father, key;
    };

    int cnt, head;
    std::vector<Node> Tree;
};
\end{lstlisting}
